

\chapter{RELATED WORK} \label{ch:related_works}
Recent advances in hardware capabilities and computational intelligence techniques have led to the introduction of robotics in numerous complex use cases.
Robotic agents continue to phase out human agents for tasks that are considered mundane or dangerous, or simply because a machine can provide better results for less cost.
Here we focus on the application of autonomous robots in exploration-based operations.
Examples of this are search and rescue missions to find survivors after a natural disaster such as an earthquake, and research missions to map water on the surface of Mars.
Exploration-based operations have shown promising boosts in performance through the use of autonomous agents for a few reasons.
Most notably, there are typically certain levels of hazard involved in exploration, and search and rescue that limit or prevent a human agent's performance.
In most cases, a robotic agent is less susceptible to the same environmental hazards as a human.
Use of robotics can help eliminate the risks of injury, disease, and death of humans involved in an operation.
The other advantages to using autonomous robots is the diverse number of sensors that a robotic agent can use, as well as their ability to analyze data quickly and without bias.
Sensors, such as an infrared camera, can accurately collect data that humans do not have the capability to observe themselves.
Large amounts of sensor data can also be processed and analyzed by a computer more efficiently than a human.
This supports the idea that a robotic agent can operate at higher performance levels than a human could in many exploration-based operations.

The basics of modeling and controlling agents within an environment were studied in the works of Poole et~al.~\cite{poole_artificial_2010}, LaVelle~\cite{lavalle_planning_2006}, and Sutton et~al.~\cite{sutton_reinforcement_1998}.
Using this platform of knowledge, we next reviewed recent implementations of autonomous control designed for real-world environments.
There is an abundant amount of existing research on the use of autonomous robots for exploration.
Both intelligent hardware and software approaches have been researched to create robotic agents that can safely and efficiently navigate in an environment.
Many of the hardware focused solutions explore clever designs for mobility features that enable an agent to traverse hazardous and complex environments~\cite{kossett_robust_2011, hopkins_survey_2009, haldane_animal-inspired_2013, hoover_bio-inspired_2010, latscha_design_2014, clark_evolving_2017, smith_tri-wheel:_2015, clark_toward_2006}.
Software solutions typically focus on the application of AI control schemas to plan hazard avoidance, and maximize efficiency~\cite{christensen_multi-robot_2017, tai_autonomous_2017, stachniss_exploration_2004, clark_mobile_2007, perea_strom_robust_2017, fink_tier-scalable_2007, bai_toward_2017}.
Each of these autonomous exploration-based research experiments share three key components: there is a robotic agent with a set of actions that it can perform, the agent must use intelligence to navigate in an environment, and the agent is goal driven.
While these components are present in all of these experiments, there is a large variation in the use cases and control schemas.
Some are designed for mapping indoor or outdoor environments~\cite{tai_autonomous_2017,  stachniss_exploration_2004, perea_strom_robust_2017}, others focus on hazard avoidance~\cite{christensen_multi-robot_2017, fink_tier-scalable_2007}, use of multiple agents working together \cite{christensen_multi-robot_2017, clark_mobile_2007}, and operation efficiency~\cite{bai_toward_2017}.
A variety of intelligent approaches for controlling the agents range from the use of Bayesian prediction models~\cite{christensen_multi-robot_2017}, artificial neural networks~\cite{tai_autonomous_2017} and machine learning~\cite{bai_toward_2017} to name a few.
Each of these experiments focused on AI controllers that were hand crafted for a specific goal.
The SCOUt project aims to address all of these use cases for autonomous agents in a unified operation setup process and adaptive control schema.

While preliminary research did not uncover any existing work on a unified process for the setup and control of exploration-based operations, the idea of a single adaptive controller for completing multiple goals is not a new concept.
Arora et~al.~\cite{arora_approach_2017, hutter_online_2018} have created control schemas that are both adaptive to an agent's capabilities, as well as a diversity of environments.
In their research conducted in 2017~\cite{arora_approach_2017}, they use a high-level approach to generate a controller that can model and analyze scientific data in a task-based approach across multiple goals.
Their later research in 2018~\cite{hutter_online_2018} achieves efficient path planning and sensor usage policies through an adaptive Bayesian framework.
SCOUt exhibits similar functionality through its use of a memory-based learning model to plan actions that will maximize goal completion and minimize damage and energy usage for a variety of operations.
Memory-based control schemas have also been used in existing experiments for autonomous decision models.
Experiments such as~\cite{fu_genetic_2003, yi_new_2011}, used a genetic algorithm (GA) to generate decision policies through the use of existing knowledge.
Arulkumaran et~al.~\cite{arulkumaran_brief_2017} cover an approach to handling memory sets, as large pools of memory can yield more accurate results but lead to issues of increased complexity and storage requirements.
There has also been a heavy use of machine learning (ML) in the field of robotics.
Specifically, \cite{arulkumaran_brief_2017, bai_toward_2017, kiumarsi_optimal_2018} all implement reinforcement learning (RL) systems in action decision models.
SCOUt's decision model follows a similar approach to \cite{kiumarsi_optimal_2018}, using an adaptive process for choosing actions through the use of a reward system.
The reward system will generate long-term and short-term rewards for each agent's performance in an operation.
Long-term rewards reflect the overall outcome of an operation, while short-term rewards reflect the outcome of each action taken during the operation.
SCOUt uses a variation of RL by building a memory of state-action rewards (SAR) to predict and critique the performance of agents acting within an environment.


Our memory-based learning model primarily borrows concepts from both partially observable Markov decision process (POMDP) and learning classifier systems (LCS).
A POMDP is a generalized decision making process used in situations when an agent's state-space cannot be fully modeled~\cite{r_cassandra_survey_1998, shani_survey_2013}.
A state-space is a collection of finite possible configurations of a problem that could occur during a defined operation.
For example, in the game of chess, the state-space is a collection of legal game positions that could occur based on the moves that each player makes within the game.
When states cannot be fully modeled, POMDPs are applied to create a probability distribution of states that might result from the actions an agent can take.
The predicted state-space is then used in action decision models.
In this project, agents must navigate through previously unknown environments.
Because the agents set of actions and the types of features within each environment will vary from operation to operation, the state-space cannot be modeled in a finite set.
For this reason, POMDP cannot fully be apply to our problem.
However, SCOUt borrows the general process of making a probability distribution of potential rewards for each action, base on the current state of an agent.
A memory of past SARs is used to predict future rewards that an agent will receive for each valid action it could take.

LCS models combine data discovery systems with a ML component to build a set of rules that can be applied in a piecewise approach to decision making~\cite{sigaud_learning_2007}.
We see this reflected in the combination of SCOUt's state comparison system~\ref{subsec:state_comparisons} with the reward system~\ref{subsec:rewards}.
While SCOUt's decision model uses an LCS approach in regards to state comparisons, the full decision process is more broad in the sense that multiple comparisons are made to contribute in action-reward prediction based on a memory of previous SARs.
When information about the environment is discovered, the new agent state is passed through piecewise functions to compare it against states in the learned memory.
These functions use weight sets that were optimized using a genetic algorithm (discussed in section~\ref{sec:scout_controller}).
State comparisons will then help the decision model predict rewards that an agent will receive for each of the valid actions it can choose.





% \cite{christensen_multi-robot_2017} Multi-robots in hazardous areas; Use Bayesian prediction model to avoid hazards.
% \cite{tai_autonomous_2017} Indoor exploration in unknown environment using a Convolutional Neural Network.
% \cite{stachniss_exploration_2004} Combination of autonomous exploration with localization mapping.
% \cite{clark_mobile_2007} Multi-robot perimeter detection.
% \cite{perea_strom_robust_2017} Exploring and mapping unknown environments.
% \cite{fink_tier-scalable_2007} Example robotics mission that requires exploration in hazardous environments.
% \cite{bai_toward_2017} Uses supervised learning for autonomously exploration with efficient user of a single sensor.

% \cite{arora_approach_2017} Use of on board systems to model scientific data and reason path/action planning.
% \cite{hutter_online_2018} Exploration and sensor planning for scientific missions.


% \cite{arulkumaran_brief_2017} Discusses the use of Deep Reinforcement Learning for "experience-driven autonomous learning", pairing with robotics and the challenges related to the complexity of memory, sampling and computation.
% \cite{fu_genetic_2003} GA approach to decision tree building for intelligent action pattern building.
% \cite{yi_new_2011} Another GA approach to decision tree building.
% \cite{kiumarsi_optimal_2018} Very similar action reward system for machine learning using actor -> environment -> critique -> reward.
