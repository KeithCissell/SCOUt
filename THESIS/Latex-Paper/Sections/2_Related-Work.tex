

\chapter{Related Work}
% ----------------------------------------------------------------------------------------------------------------------------------------
\todo{Reword/Expand}
To begin my process, I read various research papers on autonomous vehicle exploration.
The key point I extracted from my readings were the various proposed tasks and situations that were being solved by autonomy.
From these proposed tasks, I was able to put together a broad perspective of use cases and begin to draw parallels between them all.
My end goal is to create an abstracted process for goal-oriented robots regardless of what their specific tasks were, the environment they were in, and the means in which they gathered their data.
I found these were the three cornerstones of goal-oriented robotics: given task, environmental obstacles to handle, and given capabilities.
This semester’s work lays out the foundation for my research in the possibility and ease for this three dimensional problem to be abstracted.
% ----------------------------------------------------------------------------------------------------------------------------------------

The field of autonomous robotics has recently gained a high amount of attention inside and outside of the research community.
As hardware capabilities and intelligent computational techniques have continued to advance, the use of robotics is being introduced to more complex tasks.
Robotic agents continue to phase out humans agents for tasks that are considered mundane or dangerous, as well as for performance reasons where a machine can provide better results or adequate results for less cost.
Here we will focus on the application of autonomous robots in surveillance based operations.
The primary examples of surveillance based operations that we will look at are exploration based scientific research and search and rescue settings.

These two types of operations have shown promising boosts in performance through the use of autonomous agents for a few reasons.
Most notably, there are typically certain levels of hazard involved that limit the capability of a human agent and sometimes prevent them entirely.
In the majority of cases, a robotic agent is less susceptible to the same environmental hazards as a human.
As well as objectively increasing agent durability and performance capabilities, use of robotics eliminates the risk of injury, disease and death of any human(s) involved in the operation.
The other advantages to using autonomous robots is the diverse amount of sensors that an robotic agent can use, and their ability to analyze data quickly and without bias.
Sensors, such as an infrared camera, can collect data that humans do not have the capability to observe themselves.
Large amounts of sensor data can also be processed and analyzed by a computer much quicker than humans.
This supports the idea that a robotic agent can perform at higher levels than a human when performing surveillance based operations.

% Use of autonomous robots in exploration settings
There is an abundant amount of existing research on the use of autonomous robots for exploration.


\cite{christensen_multi-robot_2017} Multi-robots in hazardous areas; Use Bayesian prediction model to avoid hazards.
\cite{tai_autonomous_2017} Indoor exploration in unknown environment using a Convolutional Neural Network.
\cite{stachniss_exploration_2004} Combination of autonomous exploration with localization mapping.
\cite{clark_mobile_2007} Multi-robot perimeter detection.
\cite{perea_strom_robust_2017} Exploring and mapping unknown environments.
\cite{fink_tier-scalable_2007} Example robotics mission that requires exploration in hazardous environments.
\cite{bai_toward_2017} Uses supervised learning for autonomously exploration with efficient user of a single sensor.


% Previous research involving adaptive, intelligent controllers for multiple tasks or sensor usage.

\cite{arora_approach_2017} Use of on board systems to model scientific data and reason path/action planning.
\cite{hutter_online_2018} Exploration and sensor planning for scientific missions.



% Approaches to creating intelligent controllers.

\cite{arulkumaran_brief_2017} Discusses the use of Deep Reinforcement Learning for "experience-driven autonomous learning", pairing with robotics and the challenges related to the complexity of memory, sampling and computation.
\cite{fu_genetic_2003} GA approach to decision tree building for intelligent action pattern building.
\cite{yi_new_2011} Another GA approach to decision tree building.
\cite{kiumarsi_optimal_2018} Very similar action reward system for machine learning using actor -> environment -> critique -> reward.


Proposed approach.
How it differs from previous research.
Thesis statement.


Paper layout?
