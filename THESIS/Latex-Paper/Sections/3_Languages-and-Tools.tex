

\chapter{Languages and Tools}

\section{Back End}


\section{Front End}


\section{Communication}
Communication between the back and front ends are setup in a client server setup.
The back end is setup as a local server that handles several specific requests.
The front end acts as the client, requesting data or passing commands to the back end server.

The SCOUt Server was created with the assistance of HTTP4S, a Scala open source library. \todo{Cite HTTP4S}
HTTP4S allows easy setup and use of an HTTP request handler.
For this project, the server runs on localhost:8080.
The localhost is then accessible by any applications running locally on the same computer via HTTP requests.
A service is setup on the back end to handle specific incoming requests.
If a request has a proper URL and request method type (Get and Post), the service calls Scala functions to attempt completion of the request, and then returns an HTTP response.

\todo{Format Server Table}
Method	Path	Response
GET	/ping	Responds “pong”
GET	/current_state	Returns JSON of Environment’s current state
POST	/new_random_environment	Takes dimentions and returns a procedurally generated Environment

The server must both receive parameters as well as send data structures to a client opperating in a different language.
For this reason, all communication is formatted into JSON data structures in the request and response bodies.
In cases that parameters are passed in with a request, the JSON must properly be constructed for the service to handle it.
