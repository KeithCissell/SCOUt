

\chapter{SCOUt}
This project explores the reliability and flexibility of using a single intelligent controller to complete surveillance based operations in diverse environments.
The Surveillance Coordination and Operation Utility (SCOUt) is used to generalize environments, agents, states and actions into simple data structures.
This data then builds a platform for creating controllers, running simulations, and easy visualization.





% ==============================================================================
% PLATFORM
% ==============================================================================
\section{Platform}
Several coding languages and libraries are used in this project to provide simple and expandable utility.
The utility is laid out in a client-server architecture to allow separation of data handling and data visualization.
The server portion provides full functionality to generate unique environments, build agents and controllers, run test operations, and collect results.
Data structures on the server side are implemented using an object-oriented approach that allows objects to be extended, while keeping a single abstract component to easily maintain the parent object's behavior.
This feature is crucial when it comes to long term code maintenance and encasement.
For example, an Element is a data structure that defines the types of data that an agent could detect in the environment.
If a future project wanted to utilize this tool with an agent that could detect Ultra Violet rays in their environment, they could create an "UltraViolet" class that extends the Element object by defining a set of predefined attributes within the class.
The "UltraViolet" class can then effortlessly be integrated with all other pre-existing data structures as it will be handled as the general Element object that it extends.
These generalized data structures also provide simple handling for the client portion of the utility.
The client portion acts as a Graphical User Interface (GUI) for requesting certain actions to be performed on the server, and visualizing the data structures used on the server.


\subsection{Simulation Back End}
The back end is written in the Scala language.
Scala is a Java based, paradigm language that combines object-oriented and functional programming methods.
Object-oriented programming provides the flexibility needed to create abstracted data structures, while functional programming provides data immutability while working with large sets of diverse data.
All data storage and manipulation takes place on the back end of the platform to ensures data consistency.
Data is exported from the back end in two scenarios.
The first scenario is when data is saved to a file for long term usage.
The second is when data is passed to the client for visualization.
In both cases, the data is first encoded into Json data structures.
Data is imported to the back end when loading a file or when a request is received from the client.
Imported data is expected as a Json object which is immediately decoded and parsed into Scala data structures before usage.
The circe Scala library \todo{Reference circe} is used for the encoding and decoding of Json data.
Circe provides encoding and decoding of Json files, and a seamless integration of Json data into the Scala language.
Communication for the SCOUt server uses the library HTTP4s \todo{Reference HTTP4s} to create a local service to handle Http requests from the front end, and return an Http response.


\subsection{Visualization Front End}
The platform's front end is built around Electron, \todo{Reference Electron} a framework that allows building a native desktop application with JavaScript, HTML and CSS.
The GUI is written using all three of these languages in addition to SVG, an XML-Based format for vector graphics.
HTML structures the page within Electron, CSS provides styling and JavaScript handles all of the logic.
A few JavaScript libraries are utilized within the framework to generate visual representation for data and communicate with the back end.
The SCOUt platform uses D3, node-fetch and JQuerry. \todo{reference Node packages}
D3 (Data-Driven Documents) is a visualization library that uses HTML, SVG and CSS to create graphical representation for data.
Node-fetch is used for Http communication with the back end through XMLHttp request and response handling.
JQuerry provides integration with Json data that is passed back and forth between the client and server, as well as several functions to simplify working with DOM elements within HTML.
To manage all of the dependencies between Electron, the languages and their libraries, Node Package Manager (NPM) \todo{Reference NPM} is used.
In addition to dependency management, NPM has packages of its own that simplify the process to compile all of your code into one file for Electron to handle.
JavaScript is compiled into a browser friendly format using the Babel package, and the resulting JavaScript is then integrated into the HTML code structure using webpack.
Our GUI can then be launched using a single NPM script which will compile all of our code into a single HTML file, launch electron and load the content.
The resulting GUI then provides an interactive platform for a user to load, create and visualize data.



% ==============================================================================
% ENVIRONMENT
% ==============================================================================
\section{Environment}
Representing any environment is a tricky process.
A simulation needs to balance simplicity and coverage when modeling an environment.
Leave out too much from the model and it won’t reflect real word scenarios.
Trying to model too much can consume time and effort that could instead be spent running real world experiments.
For this experiment, an environment is represented by a single, high level object that contains a few general attributes and a collection of low level objects within it to model a static, Cartesian grid layout of a real world environment.
The lower level objects then contain specific details about the contents within the environment.

Each \textbf{Environment} created is represented as an n x m 2D grid of uniformly sized s x s square \textbf{Cells}, whose size is specified by a scaling factor.
Along with positional data, each \textbf{Cell} contains information about the different elements and anomalies present within the s x s area it represents.
An \textbf{Element} is a generalized object that represents one specific environmental attribute such as the elevation or temperature.
An \textbf{Anomaly} represents some object present within the cell that could be of interest.
Anomalies often have an effect on element values in their surrounding area which makes them “traceable”.

class Environment {
	name: String
  height: Int
  width: Int
  scale: Double
	grid: Array[Array[Cell]]
}

\subsubsection{Cell}
A Cell holds an x and y coordinate for its position within the Environment's grid attribute.
These coordinates do not reflect the actual scaled size of the environment, only an index value for the order they appear within the 2-D array data structure.
The scale can easily be applied to the physical location of a Cell as it is a shared global attribute within the Environment object.
If an element type or an anomaly is present and falls within the area that the cell covers in the environment, it appears in their respective lists for the cell.

class Cell {
  x: Int
  y: Int
  elements: Array[Element]
  anomalies: Array[Anomaly]
}

\subsubsection{Element}
An element is any measurable attributes within an environment.
For example, temperature, elevation and decibel levels are all attributes of the environment whose values can be sampled.
Different types of elements are all generalized into one abstract trait, Element.
Element can be extended into classes that represent specific element types within the environment.
The trait has a set of defined attributes that an extended class must define to identify the element type and how it behaves.
Name and unit are used for identification and displaying the element type.
The value attribute holds a numerical value for the element.
This way, when a cell holds an Element, it can store the value for the given position.
For example, the Elevation class would store the elevation level of the area within the cell as an Element of name "Elevation" and the unit would indicate the measurement unit used.
The radial flag, lowerBound and upperBound attributes guide and limit the values that can be set for the element type.
It is mostly used when generating an environment. \todo{reference environment gen section}

trait Element {
  name: String
  value: Double
  unit: String
  radial: Boolean
  lowerBound: Double
  upperBound: Double
}

\subsubsection{Anomaly}
An anomaly is any object that may be of significance to the robot such as a human or precious mineral.
Anomalies have their own effects on elements in the environment around them.
The variety of anomalies and the effects they can have are represented as data structures that follow the same extendable trait format as Element.
An Anomaly class can have an area that extends multiple cells, but must exist in at least one cell in an environment.
Anomalies can also have effects on multiple element types in its surrounding environment, which are in a list attribute.
Each Effect class defines a "seed" Element class and a range of the effect.
The seed attribute holds a specific instance of an Element class that will represent the value of that element type in the area that the attribute exists within the environment.
The range then defines the radius of the area beyond the attribute's position that the effect will "radiate".
The term radiate is used because the effect will alter the element type's values in surrounding based upon how close they are to the source of the effect (where the anomaly exists).
For example, Human is an anomaly that takes the area of a single cell, and effects Temperature and Decibel values in their environment.
If Human is much louder than the static noise level in the environment, you will see a sharp spike in Decibel values in cells nearest the Human, and the increase in value above the environment's static level will diminish the further away you move from the Human.

trait Anomaly {
  name: String
  area: Double
  effects: List[Effect]
}

trait Effect {
  seed: Element
  range: Double
}

\subsubsection{Layer}
One last important data structure is a Layer.
While Layers are not direct members of the Environment structure, they are crucial to building and analyzing the Environment.
For this reason, Layers are only generated on demand through method calls.
A Layer acts in a similar way to an Environment’s grid, except it holds a single Element instead of a cell.

class Layer {
	length: Int
	width: Int
	layer: Array[Array[Element]]
}



% ==============================================================================
% AGENT
% ==============================================================================
\section{Agent Representation}
Agents within this experiment have a core set of attributes and abilities, along with a set of sensors and a controller.
The core attributes for an agent are health, energy level, a system clock, an internal map and its current position relative to the internal map.
Because SCOUt is focused on purely observational interactions with its environment, an agent only has two categories of actions that can be taken: movement and scanning.
The agent can attempt to move one cell at a time in any of the cardinal directions.
This allows the agent to reassess after each movement attempt.
Scanning collects information about the agent's immediate environment and updates internal map.
The list of scan actions that an agent can perform is based on the set of sensors the agent is equipped with.
The controller is in charge of analyzing the current state of the agent and deciding the next action to be performed.
This project focuses on creating a single controller (SCOUt) that is highly adaptable to wide ranges of agent configurations, environments and goals.

class Agent {
  name: String
  controller: Controller
  sensors: List[Sensor]
  internalMap: Grid[Cell]
  xPosition: Int
  yPosition: Int
  health: Double
  energyLevel: Double
  clock: Double
}


\subsection{Sensor}
Sensors are created from a single trait, similar to Elements and Anomalies.
A single instance of a sensor represents a scientific instrument that could be used to gather data measurements for a specific element type.
Each sensor defines the element type it is able to measures, the energy and time costs to perform a "scan" action and its effective range.
When performing a scan, the sensor will sweep 360 degrees around the agents location and gather data in a circular area.
The circular area is calculated with the sensors range as the radius and the agents position as the center.
The element type's values discovered will then be added to the agents internal map if it was not previously known.

class Sensor {
  elementType: String
  range: Double
  energyExpense: Double
  runTime: Double
}


\subsection{Mobility and Durability}
When an agent is created, there are several attributes that can be defined to dictate how an agent will interact with different elements in the environment.
The mobility of an agent will determine what level of movement the agent is capable of within an environment, how quickly an agent can move and how much energy is required.
For example, if a drone was defined as the agent, it would have higher mobility, but would likely sacrifice the amount of sensors that could be carried.
A wheeled robot loaded with multiple sensors would have decreased mobility, but could collect a wider variety of data.
Mobility is defined by the maximum slope an agent can climb, the minimum slope an agent can traverse before taking "fall damage", and a resistance factor.
The resistance factor ties into durability and scales the amount of "fall damage" that the agent receives.

movementSlopeUpperThreshHold: Double
movementSlopeLowerThreshHold: Double
movementDamageResistance: Double

Durability factors are also considered.
This represents an agent's versatility within an environment and is directly related to specific element types.
An example is an environment with pools of water in it.
Most robots would be damaged when emerged in water, but a robot could be designed amphibiously and would therefore be impervious to damage when in contact with water.
Like many other data structures seen, these durability factors are defined per element type.
Durability is defined by an upper and lower value threshold and a resistance factor.
The thresholds define what levels of an element type that the agent can be exposed to before it begins to damage equipment.
The resistance factor then influences how much damage the agent will take at levels exceeding the threshold.

damageUpperThreshold: Double
damageLowerThreshold: Double
damageResistance: Double


\subsection{Actions}
An agent's actions are simplified to either movement or scanning actions.
These two categories cover the exploration and research aspects that would be expected from a surveillance agent.
Actions are performed in a loop where the controller assesses the agent's state and then decides an appropriate action.

In simulation, movement is handled by changing the agent's current position to an adjacent cell in one of four direction.
Movement to an adjacent cell is denoted as "north", "south", "west" or "east" based on the orientation of the x, y grid of cells that make up the environment.
Moving a single cell at a time gives the agent the opportunity to reassess its current state before continuing movement.
Distance covered by successful movement will inherently be equal to the size of the cells within the simulated environment.
Each time an agent attempts to move to a new cell, Elevation levels will be compared between the current and new cell to check if movement is possible or if it results in damage.
After the attempt has been made, changes to health, energy level and the system clock are calculated and then updated.
If the movement action is successfully completed, the current position is also updated.

When a sensor is used, it does a full 360° sweep of surrounding cells in the environment.
This creates a search circle with radius equal to the sensor's range and the center located at the agent's current position.
For each cell that fall within this search circle, the value for the sensor's given element type is extracted.
These values are then added to the agent's internal map if they did not previously exist there.
Through repeated scanning, the agent will begin to map out its surrounding environment.
This map can be then be used by the controller to determine what actions would be most beneficial to the goal at hand.


\subsection{State Representation}
For controllers to intelligently decide when to perform what actions, they need to have sufficient data about the agent and the known surrounding environment.
The agent's position, health and energy level can easily be analyzed, but the internal map containing the known environment can become a very large data structure to analyze each time the controller has to decide upon an action.
For this reason the data structure is simplified in order to reduce memory usage and computational effort required to assess a state.
What we are left with is a minimal data structure that still contains all of the useful information necessary for a controller to make intelligent decisions.
Instead of a 2-D array of cells, the internal map is represented as a list of element states, where each element state is a summary of the data known about a specific element type.

AgentState {
  xPosition: Int
  yPosition: Int
  health: Double
  energyLevel: Double
  elementStates: List[ElementState]
}

ElementState {
  elementType: String
  indicator: Boolean
  hazard: Boolean
  percentKnownInSensorRange: Double
  northQuadrant: QuadrantState
  southQuadrant: QuadrantState
  westQuadrant: QuadrantState
  eastQuadrant: QuadrantState
}

Element states contain useful information about how the specific element type was being studied during the operation and then technical information about known values.
The indicator flag tells the controller whether this element type was being collected in order to progress the goal at hand.
If the goal was to map out the elevation, the elevation ElementState would be flagged true.
If the goal was to find a human, the Temperature and Decibel ElementStates would be marked true, as their values could help indicate the presence of the human.
The hazard flag is used to mark any element that could potentially cause harm to the agent.
For example: the presence of water, high drops in elevation and extreme temperatures could potentially cause damage, and would be flagged as hazardous.
We also track the percent of known element values that are within the range of the corresponding sensor.
Technical information of each ElementState is divided into four quadrants, where each quadrant has its own state.
Because agent movement is limited to north, south, west and east, we can collapse known information from the internal map into four quadrants \todo{Image of quadrants}.

QuadrantState {
  percentKnown: Double
  averageValueDifferential: Option[Double]
  immediateValueDifferential: Option[Double]
}

The first thing that a QuadrantState looks at is the percent of values that are already known in all the cells within the quadrant.
Then, the QuadrantState stores the average and immediate known values into two "Options".
These are denoted as Options because there are instances where none of the values within the quadrant are known.
Average and immediate values are recorded relative to the value of the current cell.
Average differential takes the difference between the current value and the average of all known values in the quadrant's cells.
Immediate differential takes the difference between the current value and the cell immediately adjacent to it.
So when considering elevation values within the north quadrant, the immediate differential would be the difference between the elevation at the agents current position and the elevation within the cell directly above the current position.


\subsection{Controller}
The design of an agent is created so that as it operates within an environment, a controller can direct it to explore and gather data while keeping track of its location and internal status.
The controller is an autonomous decision making schema for managing the movement and sensor usage.
Each action that the agent takes has an effect on its internal state.
Moving in the environment requires energy usage and can result in damage (for example falling down a cliff or driving into water).
Use of sensors also requires energy and the data gathered will be stored in the internal map.
For intelligent controllers, data collected can be used by the controller to decide the next action that should be taken.
The specific controllers used are discussed in \ref{controllers}.




% ==============================================================================
% OPERATIONS
% ==============================================================================
\section{Operations}
To explore the usefulness and robustness of the intelligent controller, many different scenarios need to be explored.
All scenarios are made up of three main components: an agent, a goal and the environment.
Different configurations of each component's variables allows us to create a vast variety of scenarios.
These three components are chosen and then fed into a simulation process called an operation. \todo{Operation Diagram}
Each operation simulates the agent attempting to complete a goal within an environment.
Interactions between the agent and its environment are played out and data will is collected to record decisions made by the agent and the outcome of each decision.
Data collection takes place on a low level and a high level during each operation.

Low level data is collected each time the agent performs an action.
- The agent's state when the action was selected
- Any changes to the agent's internal state (health or energy reduction)
- If the action performed was successful (could it move, did it have enough energy to complete the action)
- A short term reward for the outcome of the action


High level data is only collected once at the simulated operation ends.
- Status of internal variables (health and energy)
- Number of actions taken during operation
- Level of goal completion
- An overall long term reward
- Long term reward given to each action taken


\subsection{Goals}
An operation is run with a given end goal for the agent to achieve.
SCOUt is designed for the observational and exploration portion of a task.
If the entire task of a robot was to traverse a hazardous area to find a certain mineral for extraction.
SCOUt would be used to guide exploration in the environment and detect the mineral.
When the mineral is found, SCOUt's process would then be completed successfully and another process could take over for the actual extraction, or the location could be recorded and another agent could be sent in.
Following this, the SCOUt agent could continue to search for more deposits of the mineral or return to base.
Using this distinction, we classify goals into two catagories for testing SCOUt's exploration and observation abilities, anomaly searching and element mapping.
Anomaly searching requires an agent to find a given anomaly within an environment.
This tests SCOUt's ability to use environmental clues to track down the anomaly.
For example, if the agent was looking for a human after a natural disaster, it could use data such as temperature readings and decibel readings to track down the person.
Element mapping is fairly strait forward.
The agent must map out as much readings of a certain element as possible.
This gives SCOUt the opportunity to discover correlations between the element is searching for and other elements present in the environment.
For example, you are move likely to find water in a valley rather than a hill.


\subsection{Rewards}
In addition to goal completion, an agent also has to be observant of its health and energy.
The more efficiently an agent can complete an objective the better.
The overall performance of an agent is measured by both its ability to complete the task at hand and the efficiency of the actions taken.
These performance measurements are calculated on a short and long term basis, and come in the form of "rewards".
In addition to measuring the performance of an agent, rewards can be used as a learning metric for an intelligent agent.
When an intelligent agent finds itself in a similar state as before, it can look at the action it took in the past and the reward that was given to decide if it should perform that action again.

\subsubsection{Short Term Rewards}
\todo{Add STS Equation}
Short term rewards are given each time an agent performs an action to reflect the immediate outcome of the action.
Energy and health depletion are major factors in this reward.
If the action required an excessive amount of energy or resulted in damage to the agent, the reward is decreased.
Other factors that come into play depend on the specific action taken.
If the agent attempted to move to a new area and fail entirely to move (a hill was too steep to climb) a deduction is made.
A small increase in reward is also applied if the agent moves into an unexplored area.
If the agent uses a scanner, the reward is adjusted to reflect the amount of new data learned.
This penalizes the agent from using a scanner twice in a row or after small movements as it is not efficient use of energy.

\subsubsection{Long Term Rewards}
Long term rewards are calculated once the operation is over.
This could mean that the agent has successfully completed its goal, or it is depleted of health or energy.
To reflect these scenarios, the reward is determined by the goal completion, remaining health and remaining energy. \todo{Add LTS Equation}
Even if a goal is completed, the agent could receive a low score if it was "reckless" and took lots of damage or used large amounts of energy.
The long term reward is then propagated backwards through all the actions that were taken.
The actions performed immediately before the end of the operation are given highest score.
Previous actions then receive diminishing reward based on \todo{LTS reward distribution equation}.



% ==============================================================================
% ENVIRONMENT GENERATION
% ==============================================================================
\section{Environment Builder}
Various environments must be modeled for simulating how an agent's controller would perform for a given task.
The SCOUt environment build tool captures important details that are necessary for agent-environment interactions, while remaining simple to implement and understand.
The tool is highly abstracted so that more details can easily be added as needed while still maintaining a defined build process.
Environments are procedurally gerated based upon a collection of parameters called an environment template.
These templates require minimal input from the user while providing a dynamic range of possible creation.
An Environment can be tweaked or even built entirely by hand, but the procedural generation process removes this overhead.

Procedural Generation Process
\begin{enumerate}
  \item Environment Template is passed in
  \item	Builder initializes a grid of empty cells
  \item	ElementSeeds are used to populate each present element type into the gird of cells
  \item	Terrain modifications are applied to manipulate their related element(s)
  \item	Anomalies are placed randomly within the environment
  \item	Anomaly effect(s) are applied to corresponding element(s) in neighboring cells
\end{enumerate}


\subsection{Environment Templates}
An Environment Template holds all the necessary data to build a specific environment.
Each template created will act as a guide in the creation of an instance of the Environment class.
A template can create similar, but unique environments each time it is used as the environment builder's guide.
This allows testing and training agent controllers multiple times in similar conditions, while still providing a dynamic range of scenarios that it may face in each environment.
Each template is comprised of the name, dimensions and scale of the environment along with lists of element seeds, terrain modifications and anomalies.

\todo{format}
class EnvironmentTemplate (
  name: String
  height: Int
  width: Int
  scale: Double
  elementSeeds: List[ElementSeed]
  terrainModification: List[TerrainModification]
  anomalies: List[Anomaly]
)


\subsection{Element Seeds}
The environment builder begins by procedurally generating one Layer of Elements at a time.
Each Element object has a companion object called an ElementSeed which holds parameters used to produce a Layer of its element type and a unique function defining how procedural generation will take place to produce the Layer.
The generation of each Layer is modeled on how the element type's values may vary in a real-world scenario.
Parameters within each Seed are set to a default value that can also be defined by creating a new instance of the ElementSeed to change how the values within the Layer will vary.

trait ElementSeed (
  elementType: String
  function buildLayer(height, width, scale)
)

The environment builder will use each ElementSeed to produce a Layer of Elements.
Each resulting Layer will be temporarily stored in a list for the remainder of the build process so they can be easily manipulated before being stored into their corresponding Cells within the Environment grid.
For an example, let’s look at the ElementSeed for producing the Elevation Layer.

\todo{format}
Class ElevationSeed (
	elementType: String = "Elevation"
  average: Double = 0.0
  deviation: Double = 1.0
) {
  function buildLayer(height, width, scale) {
    val layer: Layer = new Layer(AB.fill(height)(AB.fill(width)(None)))
    for {
      x <- 0 until height
      y <- 0 until width
    } {
      val value = randomDeviation(average)
      layer.setElement(x, y, new Elevation(value))
    }
    layer.smoothLayer(3, 3)
    return layer
  }

  function randomDeviation(average) {
    val lowerBound = average - deviation
    val upperBound = average + deviation
    return randomDouble(lowerBound, upperBound)
  }
}


The ElevationSeed’s buildLayer algorithm first initializes an empty Layer.
Next, it sets each (x,y) coordinate within the Layer to an Elevation object with a random value between a standard deviation of the average value provided [(average - deviation), (average + deviation)].
Once every Cartesian position has been set to an Elevation object with an assigned value, the layer is then smoothed.
Smoothing is a function defined within the Layer class that will reduce strong variations of Element values within the Layer.
For Elevation, this would equate to transforming a highly rigid surface into a smoother, more natural surface.

\todo{smoothing algo}

Most layer generators follow this same pattern.

\todo{format}
1.	Receive a Seed
2.	Initialize an empty Layer
3.	Initialize all Element values
4.	Smooth the Layer

Some layers are far easier to generate than others.
Latitude and Longitude layers can simply be generated by calculating the distance each cell is from the origin point on the Environment grid (cell (0,0)).


\subsection{Terrain Modifications}
Terrain modifications are provided by the user to influence the basic landscape of the environment.
Each modification represents a sevear alteration of one or more element type Layers within the environment.
Following a similar process laid out by Doran and Parberry\ref{doran_controlled_2010}, desired alterations are incorporated into the environment while still allowing unique variations of each alteration to develop.
Their controlled procedural generation process is used to produce landmasses that potentially have bodies and channels of water.
SCOUt's environment builder generalizes this process and extends it to allow multitudes of element types to be modified.
The TerrainModification trait provides an extendable template for all types of modifications that can be applied.

trait TerrainModification (
  name: String
  elementTypes: List[String]
) {
  function modify(layers: List[Layer])
}

After each ElementSeed has produced a layer for the Environment, TerrainModifications are applied one after the other, taking care not to overlap modifications (for example, we wouldn't want a hill to overlap with a valley and cancel each other out).

For an example, lets look at Elevation again.

\todo{format}
class ElevationModification (
  name: String = "Elevation Modification"
  elementType: List[String] = List("Elevation")
  modification: Double
  deviation: Double
  coverage: Double
  slope: Double
) {
  def modify(layer: Layer, constructionLayer: ConstructionLayer) = constructionLayer.getRandomUnmodified() match {
    case None => // No unmodified cells
    case Some(startCell) => {
      // Set local variables
      var modifiedCells: AB[(Int,Int)] = AB()
      val numCellsToMod = Math.round(coverage * constructionLayer.cellCount).toInt
      // Initial modification
      val startX = startCell._1
      val startY = startCell._2
      layer.setElementValue(startX, startY, modification)
      constructionLayer.setToModified(startX, startY, "elevation")
      modifiedCells.append((startX, startY))
      // Move to random, unmodified neighbors and modify
      for (i <- 0 until numCellsToMod) constructionLayer.getNextUnmodifiedNeighbor(modifiedCells) match {
        case None => // No neighbor cells to modify
        case Some((x,y)) => {
          val currentValue = layer.getElementValue(x, y).getOrElse(0.0)
          val mod = randomDouble((modification - deviation), (modification + deviation))
          val newValue = currentValue + mod
          layer.setElementValue(x, y, newValue)
          constructionLayer.setToModified(x, y, "elevation")
          modifiedCells.append((x, y))
        }
      }
      // Apply sloping factor to modified area through smoothing
      val effectedRadius = Math.abs(Math.round(modification / slope).toInt)
      for (i <- 0 until modifiedCells.length) {
        val randomIndex = randomInt(0, modifiedCells.length - 1)
        val c = modifiedCells.remove(randomIndex)
        val originX = c._1
        val originY = c._2
        for {
          x <- (originX - effectedRadius) to (originX + effectedRadius)
          y <- (originY - effectedRadius) to (originY + effectedRadius)
          if dist(x, y, originX, originY) != 0
          if dist(x, y, originX, originY) <= effectedRadius
        } layer.smooth(x, y, 2, dist(originX, originY, x, y))
      }
    }
  }
}

Here we have an ElevationModification which will allow us to create hills and valleys within the environment.
Again following the approach of \ref{doran_controlled_2010}, random, unmodified (x,y) positions are selected from the Layer to begin with and updates its value by the specified modification value.
The modifier then performs "walks" to random, unmodified neighboring Elements, updating their values by a standard deviation of the specified modification value by the deviation provided.
These walks continue until the specified coverage area has been modified, or until there are no neighboring cells that can be modified.
A special Layer smoothing is then applied to the modified Elevation values in the modified area as well as the immediate surrounding area to reduce rigidity and give a more natural change in values between neighboring cells.
This type of smoothing applies a given slopping factor within the modified area, allowing the ElevationModification to generate gentle hills or valleys or sharp cliffs depending on the slope defined.


\subsection{Anomaly Placement}
Once all TerrainModifications have been applied, anomalies are placed into the environment.
Each specified anomaly is randomly placed into cell(s) in the Environment.
For anomalies that occupy more than one cell, neighboring cells are chosen at random until the Anomaly's coverage area is met, or there are no neighboring cells that can contain the Anomaly.
Each cell containing an anomaly is updated so that the given anomaly appears in the Environment anomalies list for reference.
After an anomaly has been placed, each of the anomalies Effects are applied to update Element values in the corresponding Layers.
An Effect will alter the Element values for the occupied and surrounding positions in the effected Layer.
These alterations are typically applied as a "radiation".
For example, a "Human" anomaly might radiate heat and sound.
To account for this, a radiation function \todo{radiation function} is applied to the temperature and decibel values of cells in the effected radius.

Now that all ElementSeeds, TerrainModifications and Anomaly placements have occured, the resulting Layers containing thier respective Elementss are populated into their corresponding (x,y) Cell location within the Environment grid.
The resulting instance of an Environment class is then returned by the builder to the requesting party.



% ==============================================================================
% VISUALIZATION TOOL
% ==============================================================================
\section{Visualization Tool}
The environment build tool provides a Graphical User Interface (GUI) for creating and visualizing environments.
Electron \todo{Possibly cite Electron} is used to simulate a web page contained within a standalone desktop application.
This allows the front end to be written in JavaScript, HTML and CSS and handle communication to the back end via http over a localhost network.
Scala library http4s \todo{Possibly cite http4s} is used to create a server on a localhost network for handling the http requests from the front end.
This architecture allows all data creation and manipulation to be isolated in Scala on the back end, while allowing user interactions with the data to take place on the front end.
Launching the app starts up the Scala server in a new terminal and opens the Electron window which will begin attempts to establish communication with the server.


\subsection{Home Page}
Once connection between the server and GUI have been established, the user can choose to generate a random environment, build a custom environment, load in an environment or view an operation.
For a random environment, the user only inputs the name and n-by-m size of the environment and all other variables are selected by the server.
Building a custom environment steps the user through a series of form pages to create an environment template.
Loading an environment allows the user to select a saved environment or a saved template to use.
Selecting an operation will load the environment and log of all actions taken by an agent during a specific operation run that is saved in memory.
Once an environment has been generated and/or loaded by the server, it is returned to the GUI to be displayed.
The environment build tool will parse the returned environment into a visually meaningful data representation, with interactive capabilities to explore the specific variables within the environment.
In the case that the user selected an operation, the user will additionally be able to step through the action log of the agent during its operation.


\subsection{Template Forms Page}
\todo{add photos}
To create a template, the user will be presented with a series of form input form fields.
The forms are generated based on the available Element, TerrainModification and Anomaly classes that are defined in SCOUt's back end.
The first form will ask the user which element types, terrain modification and anomaly types it would like to include.
Some elements (environment, latitude and longitude) are required in all environments.
As the user selects what will be present in the environment, more forms will be generated for them to provide seed data (for Elements and Anomalies), or parameter definitions (for TerrainModifications).
Each form within this process will save the user’s input data on the front end.
This allows the user to go back and edit form data while moving through the different form pages, as well as go back and edit the form data when returning from the \ref{Visualizer Page}.
Form data is also set within required bounds and checked before submission.
Once the user has filled out all required form info, they can review their entire form entry from a single page and then submit it.
When submitted, the front end data is converted into Json data and sent in a request to the SCOUt server via a JavaScript fetch request.
An environment is then built based on this template on the back end and returned to the front end to be loaded into the \ref{Visualizer Page}.


Template Form Sections
\begin{enumerate}
  \item Environment Name and Size
  \item Element Types Present
  \item Terrain Modifications Present
  \item Anomalies Present
  \item Element Seeds
  \item Terrain Modification Templates
  \item Anomaly Seeds
\end{enumerate}


\subsection{Visualization Page}
\todo{add photos}
The visualization page provides an interactive overview of any given environment.
The main focus is on the display section where the entire environment grid is represented using heatmaps.
Different element type layers can be viewed independently, anomaly locations can be highlighted and specific element type values of a single cell can be viewed.
A main menu is also present to allow a user to perform higher level actions.
All of these interactive features are controlled by buttons, toggles and radio buttons within different sections the visualization page.
The primary use of the visualization page is for creating environment templates and for debugging.
Debugging usage ranges from analyzing an environment that was used for testing an agent-controller setup, to adding new features to the SCOUt platform.
New features can be adding new classes such as a new Element type, or improving the process in which environments are generated and stored in the back end.


\subsubsection{Main Menu}
The main menu provides high level functions to perform while using the environment build tool.
The main menu options are displayed at the top of the visualizer as a series of buttons.
The user can select buttons to return to the home page, regenerate the current environment (if an environment template is being used), save the current environment that is displayed, or save the current template being used (if a template is being used).
If the user wants to tweak the current template that is in use, they can simply return to the ref{Home Page} via the home button, and choose "Custom Environment" again.
Their previous form data will be saved in memory during their session.


\subsubsection{Display}
The display is laid out in a grid of cells corresponding to the environment's cell grid.
A display layer is created for each element type present within the environment using the D3 library.
D3's heatmap creates a graphical representation of data values over a 2-dimentional space, providing a solution for visually differentiating each cell's value within the environment cell grid.
Heatmaps are displayed where a higher value is indicated by a darker section in the map.
For example, when viewing the elevation's heatmap, a hill will appear darker than a valley.
The user also has the ability to select individual cells by clicking on their region in the displayed cell grid.
The display will highlight a cell once it has been selected and then load the cells data into the \ref{Legend}.


\subsubsection{Tool Bar}
The tool bar is divided into three subsections: Toggle Layers, Current Layer and Current Anomaly.
The Toggle Layers subsection provides two toggle buttons for the user to turn on and off the display of the Elevation layer and the Grid layer.
The Environment layer is a grey-scale contour map of the Elevation layer created by D3 heatmap (a contour map is the same as a heatmap, with the distinction of boarder lines between each value layer).
Because Elevation is the most fundamental piece to any environment, it is the only element type whose layer has the option of always being displayed.
The grid layer displays solid black lines between each cell within the cell grid.
The Current Layer subsection provides a list of radio buttons for all element types (excluding elevation) present within the environment.
When one of these radio buttons is selected, a green-scale, transparent heatmap of the selected element type will be populated into the display.
This element type layer will be displayed on top of the Elevation layer (if Elevation is toggled on).
Only one element type layer can be viewed at a time to prevent crowding the display.
The Current Anomaly subsection is also a set of radio buttons for each anomaly type present in the environment.
Selecting one of these will highlight all cells containing the given anomaly type in red.
Just as the case with element type layers, only one anomaly type can be viewed at a time.


\subsubsection{Legend}
The legend provides an overview of the environment in three main subsection: environment, layer and cell.
For the environment subsection, the name of the current environment is displayed along with the dimensions and minimum and maximum elevation within the environment.
The layer subsection displays the minimum and maximum values of the selected element type layer that is being displayed, as well as the value at the selected cell (if a cell is selected).
When a cell is selected, the values of all element types of the given cell are presented in a list, as well as the cell's relative coordinates.


\subsubsection{Operation Log}
The operation log section is a special section that only appears in the visualizer when the user loads an operation run.
This section has buttons that allow the user to step through each event that took place during the given agent's operation.
The user can select to step forward or backwards by 1 or 10 events.
When each event is loaded into the visualizer, the display section will update by selecting the cell where the agent is currently located.
The operation section will display text that shows what number event is currently being viewed, the action that was chosen, the health and energy of the agent during this event, and the rewards that were received.
