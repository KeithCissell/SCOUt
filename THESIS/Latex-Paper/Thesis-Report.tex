\RequirePackage[l2tabu, orthodox]{nag}

\documentclass[12pt]{report}
\usepackage[left=1.5in, right=1in, top=1in, bottom=1in]{geometry}
\usepackage{mathptmx}
\usepackage{setspace}
\doublespacing

%%% Include todo notes while writing draft (\listoftodos, \todo, \missingfigure)
\usepackage[color=white]{todonotes}

%%% Subliminal refinements towards typographical perfection (1%)
\usepackage[stretch=10]{microtype}

%%% Handle input and output of accented/special characters and modern fonts
\usepackage[T1]{fontenc}
\usepackage[utf8]{inputenc}
\usepackage{lmodern}
\usepackage[american]{babel}
\usepackage{csquotes}
\usepackage{cite}
\usepackage{graphicx}
\usepackage{listings}
\usepackage{amsmath}
\usepackage{amsfonts}
\usepackage{float}
\usepackage{algorithm,algorithmic}

\usepackage[toc,page]{appendix}

% Equation captions
\usepackage{caption}
\DeclareCaptionType{capeq}
\captionsetup[capeq]{name=Equation, labelformat=simple, labelsep=colon}
% Code listings
\captionsetup[lstinputlisting]{name=Code, labelformat=simple, labelsep=colon}
% Appendix items
\DeclareCaptionType{appx}
\captionsetup[appx]{name=Appendix, labelformat=simple, labelsep=colon}
% tables
\DeclareCaptionType{tblcap}
\captionsetup[tblcap]{name=Table, labelformat=simple, labelsep=colon}


% Code settings
\lstset{
  language=Scala, % C, C++, Java, SQL are from the around hundred available
  basicstyle=\ttfamily,
  numbers=left,
  numberstyle=\footnotesize,
  stepnumber=1,
  numbersep=2.0mm}


% JSON DEF
% \colorlet{punct}{red!60!black}
% \definecolor{background}{HTML}{EEEEEE}
% \definecolor{delim}{RGB}{20,105,176}
% \colorlet{numb}{magenta!60!black}
%
% \lstdefinelanguage{json}{
%     basicstyle=\normalfont\ttfamily,
%     numbers=left,
%     numberstyle=\scriptsize,
%     stepnumber=1,
%     numbersep=8pt,
%     showstringspaces=false,
%     breaklines=true,
%     frame=lines,
%     backgroundcolor=\color{background},
%     literate=
%      *{0}{{{\color{numb}0}}}{1}
%       {1}{{{\color{numb}1}}}{1}
%       {2}{{{\color{numb}2}}}{1}
%       {3}{{{\color{numb}3}}}{1}
%       {4}{{{\color{numb}4}}}{1}
%       {5}{{{\color{numb}5}}}{1}
%       {6}{{{\color{numb}6}}}{1}
%       {7}{{{\color{numb}7}}}{1}
%       {8}{{{\color{numb}8}}}{1}
%       {9}{{{\color{numb}9}}}{1}
%       {:}{{{\color{punct}{:}}}}{1}
%       {,}{{{\color{punct}{,}}}}{1}
%       {\{}{{{\color{delim}{\{}}}}{1}
%       {\}}{{{\color{delim}{\}}}}}{1}
%       {[}{{{\color{delim}{[}}}}{1}
%       {]}{{{\color{delim}{]}}}}{1},
% }



\begin{document}

\title{An Adaptive Memory-Based Reinforcement Learning Controller}
\author{Keith August Cissell}
\date{June 2018}

\maketitle




\begin{abstract}
  % ----------------------------------------------------------------------------------------------------------------------------------
  \todo{Reword/Expand}
  The purpose of this project is to create an artificially intelligent "mind" to make observational decisions for a mobile robot.
  The mobile robot will have a set of sensors with which it can survey its surrounding environment. It will store this observed data, perform an analysis and plan its next move accordingly.
  The plan is to drop a robot into unfamiliar environments with a set goal to achieve and let it learn on its own the best way to approach the given situation.
  Usages can be as simple as mapping out an unknown area, to as difficult as searching for survivors after a natural disaster.
  The robot will have internal and external limitations it must work with such as battery life and terrain obstacles.
  The goal is to create AI that can learn how to use its observational skills to achieve a goal within a dynamic environment.
  % ------------------------------------------------------------------------------------------------------------------------------------
\end{abstract}

% 

\section{Acknowledgement}
I would like to thank the entire Computer Science department at Missouri State University, specifically my committee members, Dr. Anthony Clark, Dr. Lloyd Smith, and Dr. Kenneth Vollmar.
A special thanks to Dr. Anthony Clark for advising and assisting me every step of the way, I truly could not have accomplished this without him.

I dedicate this paper to my parents who have always encouraged me to strive for greatness.
Their support in both my life and quest for knowledge cannot be put into words.


\tableofcontents
\newpage



\chapter{Introduction}
As research in the fields of autonomous systems and robotics have become more extensive, it is evident that there are a wide range of application for robots with integrated autonomy.
There are rovers, drones and even aquatic robots that are capable of decision making in their own environments.
The tasks that these robots carry out can greatly vary as well.
This variance can cause a demand for distinct software and hardware to achieve each robot’s given task.
However, almost all autonomous robots operate similarly through their use of observation (typically with external sensors) and analytics of the data that is observed.

A great deal of research has been done in hybrid robots and creating hardware that is multifunctional to various tasks
However, there is not an extensive amount of research on software with the capability to integrate with multiple robot compositions and tasks.
Most of this is due to the fact that each robot has unique capabilities that do not overlap with many other robots.
Autonomous robots seem to focus in on a certain niche and require their systems to be built from the ground up each time.
This leaves the question of what pieces of autonomous control can be abstracted.

There are many evolutionary computing approaches that can be applied to decision making processes.
These methods are commonly used in situations when there are a known number of controllable variables and a wide solution space to be explored.
This makes them great candidates for creating a system which drives the decision-making process of autonomous robots.
In particular, neural networks and deep neural networks trained in simulations seem to be a promising architecture for finding optimal control patterns in the diverse applications of autonomous robots.

This project approaches the problem from the bottom up.
It looks at the very basics of autonomous robotics.
This is: the collection of data from sensors, analytics of incoming data, and the output of response controls.
Additionally, these three steps are repetitively being performed to achieve a given objective.
I have broken this project into three phases.
The first phase involves setting up a simulation environment to be used for training the autonomous system.
Next, a graphical interphase will be integrated with the simulation data to allow for easy debugging.
Finally, an Artificially Intelligent system will be trained to take in various sets of environmental data as inputs, make decisions based on these inputs and its current objective, and produce a response.

The project is still a work in progress and this paper will only present phases one and two.
These phases cover the procedural environment generator and the graphical interface that pairs with it.
The implementation of the abstracted autonomous system will come in future work.
The main topic covered looks at the representation and formulaic production of data that will be used to represent an environment.
The graphical interfaces capabilities will also be touched on.
For the AI component, we will look at the various evolutionary methods that hold great potential for our given problem setup.



\chapter{Related Work}
The field of autonomous robotics has recently gained a high amount of attention inside and outside of the research community.
As hardware capabilities and intelligent computational techniques have continued to advance, the use of robotics is being introduced to more complex tasks.
Robotic agents continue to phase out humans agents for tasks that are considered mundane or dangerous, as well as for performance reasons where a machine can provide adequate or better results for less cost.
Here we will focus on the application of autonomous robots in surveillance based operations.
The primary examples of surveillance based operations that we will look at are exploration based scientific research and search and rescue settings.

These two types of operations have shown promising boosts in performance through the use of autonomous agents for a few reasons.
Most notably, there are typically certain levels of hazard involved that limit the capability of a human agent and sometimes prevent them entirely.
In the majority of cases, a robotic agent is less susceptible to the same environmental hazards as a human.
As well as objectively increasing agent durability and performance capabilities, use of robotics eliminates the risk of injury, disease and death of any human(s) involved in the operation.
The other advantages to using autonomous robots is the diverse amount of sensors that an robotic agent can use, and their ability to analyze data quickly and without bias.
Sensors, such as an infrared camera, can collect data that humans do not have the capability to observe themselves.
Large amounts of sensor data can also be processed and analyzed by a computer much quicker than humans.
This supports the idea that a robotic agent can perform at higher levels than a human when performing surveillance based operations.

% Use of autonomous robots in exploration settings
There is an abundant amount of existing research on the use of autonomous robots for exploration
\cite{christensen_multi-robot_2017, tai_autonomous_2017, stachniss_exploration_2004, clark_mobile_2007, perea_strom_robust_2017, fink_tier-scalable_2007, bai_toward_2017}.
Each of these autonomous exploration based research experiments share three key components.
There is a robotic agent with a set of actions that can be performed, the agent must use intelligence to navigate their environment, and the agent is goal driven.
While these three components are present in all of their experiments, there is a large variation in the uses and control schemas.
Some are designed for mapping indoor or outdoor environments \cite{tai_autonomous_2017,  stachniss_exploration_2004, perea_strom_robust_2017}, others focus on hazard avoidance \cite{christensen_multi-robot_2017, fink_tier-scalable_2007}, some use multiple agents working together \cite{christensen_multi-robot_2017, clark_mobile_2007}, and some focus on operation efficiency \cite{bai_toward_2017}.
A variety of intelligent approaches for controlling the agent(s) range from the use of Bayesian prediction models \cite{christensen_multi-robot_2017}, Neural Networks \cite{tai_autonomous_2017} and machine learning \cite{bai_toward_2017} to name a few.
Each of these experiments were hand crafted for their specific goal at hand.
The SCOUt project aims to encompass all of these use cases for autonomous agents through the use of one universal process and control schema.
% \cite{christensen_multi-robot_2017} Multi-robots in hazardous areas; Use Bayesian prediction model to avoid hazards.
% \cite{tai_autonomous_2017} Indoor exploration in unknown environment using a Convolutional Neural Network.
% \cite{stachniss_exploration_2004} Combination of autonomous exploration with localization mapping.
% \cite{clark_mobile_2007} Multi-robot perimeter detection.
% \cite{perea_strom_robust_2017} Exploring and mapping unknown environments.
% \cite{fink_tier-scalable_2007} Example robotics mission that requires exploration in hazardous environments.
% \cite{bai_toward_2017} Uses supervised learning for autonomously exploration with efficient user of a single sensor.

While my research did not uncover any existing work on a universal process for project setup, the idea of a single adaptive controller for completing multiple goals is not a new concept.
Other researchers \cite{arora_approach_2017, hutter_online_2018} have created control schemas that are both adaptive to an agent's capabilities as well as varieties of environments.
In \cite{arora_approach_2017}, they use a high level approach for a controller to model and reason with scientific data to generate task based approaches to multiple goals.
Research in \cite{hutter_online_2018} achieves efficient path planning and sensor usage policies through an adaptive Bayesian framework.
SCOUt's intelligent controller is used to achieve similar functionalities as both of the above systems.
% \cite{arora_approach_2017} Use of on board systems to model scientific data and reason path/action planning.
% \cite{hutter_online_2018} Exploration and sensor planning for scientific missions.

The control schema for SCOUt uses memory based reinforcement learning to decide the best possible action to take in a given state to minimize damage to the agent, energy usage and time elapse.
Memory based control schemas have been used in existing experiments \cite{fu_genetic_2003, yi_new_2011} for decision based models.
Both of these experiments used a genetic algorithm (GA) to generate decision policies through the use of existing knowledge (memory).
Reinforcement learning is heavily used in the field of robotics.
Specifically, \cite{arulkumaran_brief_2017, bai_toward_2017, kiumarsi_optimal_2018} all implement reinforcement based learning for action decision models.
\cite{arulkumaran_brief_2017} also covers the handling of large memory sets as growing the memory pool in size can yield more accurate results, but lead to greater complexity and the memory storage management issues involved.
SCOUt's decision schema follows a similar model to \cite{kiumarsi_optimal_2018}, using an adaptive process of choosing an action based on a reinforcement learned score system, critiquing the actors performance in an environment, and updating the action selection process.
% \cite{arulkumaran_brief_2017} Discusses the use of Deep Reinforcement Learning for "experience-driven autonomous learning", pairing with robotics and the challenges related to the complexity of memory, sampling and computation.
% \cite{fu_genetic_2003} GA approach to decision tree building for intelligent action pattern building.
% \cite{yi_new_2011} Another GA approach to decision tree building.
% \cite{kiumarsi_optimal_2018} Very similar action reward system for machine learning using actor -> environment -> critique -> reward.



\chapter{SCOUt}
This project explores the reliability and flexibility of using a single intelligent controller to complete surveillance-based operations in diverse environments.
The Surveillance Coordination and Operation Utility (SCOUt) is used to generalize environments, agents, states and actions into abstract data structures.
This data then builds a platform for creating controllers, running simulations, and visualizing outputs.



% ==============================================================================
% PLATFORM
% ==============================================================================
\section{Platform}
Several coding languages and libraries are used in this project to provide a simple and expandable platform.
It is laid out in a client-server architecture to allow separation of data handling and data visualization.
The server portion provides full functionality to generate unique environments, build agents and controllers, run test operations, and collect results.
Data structures on the server side are implemented using an object-oriented architecture of traits, classes and class instances.
Traits are abstract objects that can be inherited by multiple classes.
Each class that inherits a trait can add specific values and behaviors to the object.
Different instances of a class can be declared for repeated usage within code.
This feature is crucial when it comes to long term code handling and maintenance.
All classes that inherit from the same trait can be handled using the same logic, yet each class can behave in a unique way.
For example, Element is a trait that defines different types of data that an agent could detect in the environment.
Elevation is a class that inherits the Element trait, and a specific instance of Elevation can be created for an area that has an elevation level of 100 feet.
The trait-class-instance architecture also simplifies the addition of new class definitions.
If a future project wanted to utilize the SCOUt platform with an agent that could detect ultra violet rays in their environment, they could define a new "UltraViolet" class that inherits the Element trait.
The "UltraViolet" class can then effortlessly be integrated with all other pre-existing code since it will be handled as the general Element triat that it extends.
This architecture's usefulness extends into the client portion of the platform as well.
The SCOUt client is a Graphical User Interface (GUI) for requesting actions to be executed by the server and visualizing the data structures returned.
Because the majority of the data structures used in this platform are abstracted, the front end can be generalized to handle any new classes created without any maintenance required for the GUI.


\subsection{Simulation Back End}
The back end is written in the Scala language.
Scala is a Java based, paradigm language that combines object-oriented and functional programming methods.
Object-oriented programming provides the flexibility needed for the trait-class-instance architecture, while functional programming provides immutability when working on large sets of diverse data.
All data storage and manipulation takes place on the back end of the platform to ensures consistency.
Data is only imported or exported on the back end in two scenarios: file storage and client-server communication.
In both cases, it is assumed that immutability is maintained.
File storage is the only case where data is open to manipulation outside of the back end.
Client-server communication only allows variables to be passed into the back end via requests, and a copy of data structures are returned to the front end for visualization only.
Any alterations on the front end will have no effect on the original copy on the back end.
To allow storage and communication, the back end encodes and decodes data into Json objects.
When imported, Json objects are immediately decoded and parsed into Scala data structures before usage.
The circe Scala library \todo{Reference circe} is used for the encoding and decoding of Json data.
Circe provides integration of Json objects in the Scala language to allow seamless encoding and decoding.
Communication for passing and receiving Json objects between the front and back end is achieved with the HTTP4s library. \todo{Reference HTTP4s}
The SCOUt server is setup using HTTP4s' blaze-server to create a local service for handling Http communication.


\subsection{Visualization Front End}
The platform's front end is built around Electron, \todo{Reference Electron} a framework that allows building a native desktop application with JavaScript, HTML and CSS.
The GUI is written using all three of these languages.
HTML structures the page within Electron, CSS provides styling and JavaScript handles all of the logic.
The SCOUt platform uses D3, node-fetch and JQuerry JavaScript libraries to assist with data visualization and communication to the back end. \todo{reference Node packages}
D3 (Data-Driven Documents) is a visualization library that uses SVG (an XML-Based format for vector graphics) to create graphical representation of data sets.
Node-fetch is used for Http communication with the back end through XMLHttp request and response handling.
JQuerry provides integration with Json data that is passed back and forth between the client and server, as well as several functions to simplify working with DOM elements within HTML.
Node Package Manager (NPM) \todo{Reference NPM} is used to maintain all of the dependencies between Electron, the three languages and the JavaScript libraries on the platform's front end.
In addition to dependency management, NPM has packages of its own that simplify the process for compiling code into a single file for Electron to handle.
The Babel \todo{ref Babel} package transpiles JavaScript into a browser friendly format, then webpack \todo{ref webpack} integrates the resulting JavaScript into the HTML code for a single JavaScript file for Electron.
The GUI can then be launched using an NPM script to compile all of the code, launch Electron and load the content.



% ==============================================================================
% ENVIRONMENT
% ==============================================================================
\section{Environment}
Modeling a real-world environment in a simulation is a tricky process.
Each model needs to balance simplicity and coverage.
If too much is left out of the model, it won’t reflect real word scenarios.
On the other hand, attempting to model too much can be impractical as it consumes effort and resources that could instead be spent running real world experiments.
For this experiment, environments are modeled as a high level class containing a collection of lower level classes that together form a cleverly simplified representation of a real world environment.

The \texttt{Environment} class holds an $n x m$ 2D grid of uniformly sized $s x s$ square cells, where $n x m$ is the total area of the \texttt{Environment} and $s x s$ is the area each \texttt{Cell} represents within the \texttt{Environment}.
Along with positional data, each \texttt{Cell} contains information about the different elements and anomalies present within the $s x s$ area it covers.
An \texttt{Element} is a generalized object that represents one specific environmental attribute, such as the elevation or temperature.
An \texttt{Anomaly} represents some object present within an \texttt{Environment} that could be of interest.
Anomalies often have an effect on element values in their surrounding area which makes them “traceable”.

\todo{add a diagram}

\begin{lstlisting}[language=Scala]
class Environment (
  name: String
  height: Int
  width: Int
  scale: Double
  grid: Array[Array[Cell]]
)
\end{lstlisting}

\subsubsection{Cell}
A \texttt{Cell} holds $x$ and $y$ coordinate for its relative position within the \texttt{Environment}'s grid.
These coordinates do not reflect the actual size of the Cell, only an index value for the order they appear within the 2-dimensional array data structure.
The \texttt{Environment}'s scale can easily be applied to the physical location of a Cell as it is a shared global attribute within the \texttt{Environment} class.
If an element type or an anomaly is present within the area that the cell covers in the \texttt{Environment} grid, it will appear in its respective list.

\begin{lstlisting}[language=Scala]
class Cell (
  x: Int
  y: Int
  elements: Array[Element]
  anomalies: Array[Anomaly]
)
\end{lstlisting}

\subsubsection{Element}
An element can be any measurable attributes within an environment.
For example: temperature, elevation and decibel levels are all attributes of the \texttt{Environment} whose values can be measured.
All element types are all generalized by the abstract trait, Element.
The trait has a set of defined attributes that an inheriting class must define to identify the element type and how it behaves.
Name and unit are used for identification and displaying the element type.
The value attribute holds a numerical value for the measurement of each instance.
For example, an instance of the Elevation class would store a measurement of the elevation level in a certain area.
The radial flag, lowerBound and upperBound attributes guide and limit the values that can be set for the element type.
These are used when procedurally generating an environment (covered in \ref{}). \todo{reference environment gen section}

\begin{lstlisting}[language=Scala]
trait Element {
  name: String
  value: Double
  unit: String
  radial: Boolean
  lowerBound: Double
  upperBound: Double
}
\end{lstlisting}



\subsubsection{Anomaly}
Anomalies are any object that may be of significance to an agent, such as a human or precious mineral.
Anomalies have their own effects on element values in the environment around them.
Like Elements, Anomalies and Effects are defined as classes that inherit from a single trait.
An Anomaly class can occupy multiple Cells, but must occupy at least one cell in an environment.
Anomalies can also have multiple Effects on multiple types of element values in surrounding Cells
These Effects are declared in a list attribute.

\begin{lstlisting}[language=Scala]
trait Anomaly {
  name: String
  area: Double
  effects: List[Effect]
}
\end{lstlisting}

Each Effect class defines a "seed" Element class and a range of the effect.
The seed attribute holds a specific instance of an Element class that will represent the value of that element type in the area that the attribute exists within the environment.
The range then defines the radius of the area beyond the attribute's position that the effect will "radiate".
The term radiate is used because the effect will alter the element type's values in surrounding based upon how close they are to the source of the effect (where the anomaly exists).

\begin{lstlisting}[language=Scala]
trait Effect (
  seed: Element
  range: Double
)
\end{lstlisting}

For example, Human is an anomaly that takes the area of a single cell, and Effects the Temperature and Decibel values in their Environment.
If the Human is much louder than the ambient noise level in the environment, there will be a sharp spike in Decibel values in Cells nearest the Human, with a diminishing increase for values in surrounding Cells.
\todo{Show example code or photo?}


\subsubsection{Layer}
One last important data structure is a Layer.
While Layers are not direct members of the Environment class structure, they are crucial to building and analyzing the Environment.
For this reason, instances of the Layer class are only generated on demand through method calls.
A Layer is designed in the same 2-dimentional structure as the Environment grid, but holds a collection of Elements instead of Cells.
\todo{Photo of Layer}

\begin{lstlisting}[language=Scala]
class Layer (
  length: Int
  width: Int
  layer: Array[Array[Element]]
)
\end{lstlisting}



% ==============================================================================
% AGENT
% ==============================================================================
\section{Agents}
Agents within this experiment have a core set of attributes and abilities, along with a set of sensors and a controller.
The core attributes for an agent are health, energy level, an internal map and its current position relative to the environment grid.
Because SCOUt is focused on purely observational interactions with its environment, an agent only has two categories of actions that can be performed: movement and scanning.
The agent can attempt to move one cell at a time in any of the four cardinal directions.
This allows the agent to reassess after each movement attempt.
Scanning collects information about the agent's immediate environment and updates internal map.
The list of scan actions that an agent can perform is based on the set of sensors the agent is equipped with.
The agent's controller is in charge of analyzing the current state of the agent and deciding the next action to be performed.
This project focuses on creating a single controller that is highly adaptable to wide ranges of agent configurations, environments and goals titled as the SCOUt controller.
This controller should be able to show adaptability when performing new tasks and when controlling different agents.
To model how an Agent will interact with an Environment, Mobility and a collection of Durability factors are defined per Agent.

\begin{lstlisting}[language=Scala]
class Agent (
  name: String
  controller: Controller
  sensors: List[Sensor]
  internalMap: Array[Array[Cell]]
  xPosition: Int
  yPosition: Int
  health: Double
  energyLevel: Double
  mobility: Mobility
  durabilities: List[Durability]
)
\end{lstlisting}


\subsection{Sensor}
Sensors are created using the same trait-class-instance architecture as Elements and Anomalies.
The Sensor class models a scientific instrument that could be used for gathering data measurements of a specific element type.
Each class defines the element type it is able to measures, the energy it costs to perform a "scan" action, its effective range, and two flags indicating if the element type being searched for if hazardous or considered an indicator.
When performing a scan, the sensor will sweep 360 degrees around the agents location and gather data within the circular area.
The circular scan area is calculated with the sensors range as the radius and the agents position as the center.
Any Element values that were previously unknown to the agent are then added to the internal map.
Hazardous elements are flagged in a sensor Class when its element type has the potential to cause harm to the given Agent.
The indicator flag is set when the element type is believed to be of importance for completing the goal.
For example, if an Agent was searching for a Human, Temperature and Decibel Sensors would be flagged as indicators because their values can potentially help lead the agent to the Human.

\begin{lstlisting}[language=Scala]
class Sensor (
  elementType: String
  range: Double
  energyExpense: Double
  hazard: Boolean
  indicator: Boolean
)
\end{lstlisting}


\subsection{Mobility}
Mobility is a stand alone class that will determine the ease and limitations of the agent moving within an environment.
For example, if a drone was modeled as the agent, it would have a higher range of mobility, but would likely sacrifice the amount of sensors that could be carried.
A wheeled robot loaded with multiple sensors would likely have decreased mobility, but could collect a wider variety of data.
Mobility is defined by maximum slope an agent can climb, the minimum slope an agent can traverse before taking "fall damage", a resistance factor and the energy cost required for movement.
The resistance factor is used to scales the amount of damage that the agent may take (for example if it fell off of a cliff).
Movement cost and slope cost used to calculate how much energy is used when attempting a movement action.
The total cost calculated is scaled based on the distance moved, and the slope of the elevation between the agent's current position and the position that it attempts to move to.


\begin{lstlisting}[language=Scala]
class Mobility (
  movementSlopeUpperThreshHold: Double
  movementSlopeLowerThreshHold: Double
  movementDamageResistance: Double
  movementCost: Double
  slopeCost: Double
)
\end{lstlisting}

\subsection{Durability}
Like many other data structures in this platform, Durability factors are defined per element type.
These factors model how an Agent will be effected by different Elements they come in contact with during an Operation.
For example, consider an environment with pools of water in it.
Most robots would be damaged when emerged in water, but an amphibious robot could be modeled to be impervious to damage when in contact with water.
Durability is defined by an upper and lower value threshold and a resistance factor.
The thresholds define what levels of an element type that the agent can be exposed to before it begins to take damage.
The resistance factor then influences how much damage the agent will take at levels exceeding the threshold.

\begin{lstlisting}[language=Scala]
class Duribility (
  elementType: String
  damageUpperThreshold: Double
  damageLowerThreshold: Double
  damageResistance: Double
)
\end{lstlisting}


\subsection{Actions}
Agents interact with the Environment via actions.
Actions are simplified to either movement or scanning actions.
These two categories cover the exploration and research aspects that required for most surveillance operations.
The Agent's Controller is in charge of deciding what action to perform.

In simulation, movement is handled by changing the agent's current position to an adjacent cell in one of four direction.
Movement to an adjacent cell is denoted as "north", "south", "west" or "east" based on the orientation of the x, y grid of cells that make up the environment. \todo{add photo of movement}
Moving a single cell at a time gives the agent the opportunity to reassess its current state before selecting the next action.
Distance covered by successful movement will inherently be equal to the size of the cells within the simulated environment.
Each time an agent attempts to move to a new cell, Elevation levels will be compared between the current and new cell to check if movement is possible or if it results in damage (based upon the Agent's Mobility).
After the attempt has been made, changes to health and energy level are calculated based upon the Agent's Durability factors and then updated.
If the movement action is successfully completed, the current position is also updated.

An Agent can perform scans of the Environment using available Sensors.
For each cell that fall within the Sensor's search radius, the value for the sensor's given element type is extracted.
These values are then added to the agent's internal map if they did not previously exist there.
Through repeated scanning, the agent will begin to map out its surrounding environment.
Data collected in the internal map can be then be used by the controller to determine what actions would be most beneficial to the goal at hand.


\subsection{State Representation}
For controllers to intelligently decide when to perform what actions, they need to have sufficient data about the agent and the known surrounding environment.
The agent's position, health and energy level can easily be analyzed, but the internal map containing the known environment is a very large data structure to analyze each time the controller has to decide upon an action.
For this reason the data structure is simplified in order to reduce memory usage and computational effort required to analyze a state.
AgentState is the minimal data structure that contains all of the useful information necessary for a controller to make intelligent decisions.
Instead of a 2-dimensional array of cells, the internal map is represented as a list of ElementStates, where each ElementState is a summary of the data known about a specific element type.

\begin{lstlisting}[language=Scala]
class AgentState (
  xPosition: Int
  yPosition: Int
  health: Double
  energyLevel: Double
  elementStates: List[ElementState]
)
\end{lstlisting}

\begin{lstlisting}[language=Scala]
ElementState (
  elementType: String
  indicator: Boolean
  hazard: Boolean
  percentKnownInSensorRange: Double
  northQuadrant: QuadrantState
  southQuadrant: QuadrantState
  westQuadrant: QuadrantState
  eastQuadrant: QuadrantState
)
\end{lstlisting}

Element states contain useful information about what information is known for a specific element type during the operation.
The indicator flag can cue the controller on whether the element type is being analyzed in order to progress the goal at hand.
If the goal was to map out the elevation levels in an Environment, the elevation ElementState would be flagged true.
If the goal was to find a human, the Temperature and Decibel ElementStates would be marked true, as irregular changes in these values could help indicate the presence of the Human.
The hazard flag is used to mark any Element that could potentially cause harm to the agent.
For example: the presence of water, large changes in elevation and extreme temperatures could potentially cause damage, and would be flagged as hazardous.
We also track the percent of known element values that are within the range of the corresponding sensor.
Technical information of each ElementState is divided into four quadrants, where each quadrant has its own state.
Because agent movement is limited to north, south, west and east, we can collapse known information from the internal map into four quadrants. \todo{Image of quadrants}

\begin{lstlisting}[language=Scala]
class QuadrantState (
  percentKnown: Double
  averageValueDifferential: Option[Double]
  immediateValueDifferential: Option[Double]
)
\end{lstlisting}

The first thing that a QuadrantState looks at is the percent of values that are already known in all the cells within the quadrant.
Then, the QuadrantState stores the average and immediate known values into two "Options".
These values are defined as Options because there are instances where the values within the quadrant are not known or may not exist (if the Agent is at the edge of the defined Environment grid).
Option is a built in Scala data type that can be None when undefined or Some(<value>) when defined.
This data type preserves data immutability as all variables must be defined within Scala code.
Average and immediate values are recorded as differentials relative to the value of the current cell.
Average differential takes the difference between the current Cell's value and the average of all known values in the quadrant's collection of Cells.
Immediate differential takes the difference between the current Cell's value and the Cell immediately adjacent to it.
So when considering elevation values within the north quadrant, the immediate differential would be the difference between the elevation at the agents current position and the elevation within the adjacent cell to the North.


\subsection{Controller}
A Controller is the goal driven decision making schema that an Agent will use when navigating an Environment.
The Agent will pass its current state to the controller, along with a list of valid actions that it can take.
The Controller will then decide the best action that can be taken given the current state.
Controllers are defined using the trait-class-instance architecture.
Inheriting Controller classes are provided a setup and shutdown function to perform any initializations or final actions once an Operation has ended.
The Controller's schema is defined within the selectAction function.
This function takes in the list of valid action and the current AgentState and must return a single action which the Agent will attempt to perform.
Specific Controllers and their schemas are analyzed and discussed in the \ref{Experiments} section.

\begin{lstlisting}[language=Scala]
trait Controller {
  def setup(mapHeight: Int, mapWidth: Int): Unit
  def selectAction(actions: List[String], state: AgentState): String
  def shutDown(stateActionPairs: List[StateActionPair]): Unit
}
\end{lstlisting}



% ==============================================================================
% OPERATIONS
% ==============================================================================
\section{Operations}
To explore the usefulness and robustness of the intelligent controller, many different scenarios need to be simulated.
All simulations are made up of three main components: an agent, a goal and the environment.
Different combinations of each component allows the creation of a large variety of scenarios.
Each simulation follows a defined process called an Operation. \todo{Operation Diagram}
Each Operation will simulate an Agent's attempt to complete a Goal within an Environment.
The Operation will record data for each event that occurs between the Agent and Environment and the final outcome.
These data collections are denoted as short term and long term events respectively.

Short term events are collected each time the agent performs an action.

\begin{itemize}
  \item The agent's state when the action was selected
  \item Any changes to the agent's internal state (health or energy reduction)
  \item If the action performed was successful (could it move, did it have enough energy to complete the action)
  \item A short term reward for the outcome of the action
\end{itemize}

A long term event is only collected once at the simulated operation ends.

\begin{itemize}
  \item Status of internal variables (health and energy)
  \item Number of actions taken during operation
  \item Level of goal completion
  \item An overall long term reward
  \item Long term reward given to each action taken
\end{itemize}


\subsection{Goals}
Because SCOUt is designed for observation and exploration, two Goal types are analyzed: anomaly searching and element mapping.
This is not say that SCOUt would be limited to Operations which only involve these types of tasks.
SCOUt is intended to be coupled with other tasks.
For example, if the entire task of a robot was to traverse a hazardous area to find a certain mineral for extraction.
SCOUt would be used to guide exploration in the environment and detect the mineral.
When the mineral is found, SCOUt's process would then be completed successfully and a separate process could take over for the actual extraction, or the location could be recorded by SCOUt and another agent could be sent in.
After the other agent or process completes its task, SCOUt could continue to search for more deposits of the mineral or return to base.
For anomaly searching goals, the agent is required to find a specified anomaly within an environment.
This tests SCOUt's ability to use environmental clues to track down the anomaly.
For example, if the agent was looking for a human after a natural disaster, it could use data such as temperature and decibel readings to locate the person.
Element mapping is fairly strait forward.
The agent is must map out as much data about the specified element type as possible.


\subsection{Rewards}
In addition to goal completion, an agent also has to be observant of its health and energy on a short term scale.
The more efficiently an agent can complete an Operation the better.
The overall performance of an agent is measured by both its ability to complete the task at hand and the safety and efficiency of the actions taken.
These performance measurements are calculated on a short and long term basis, and come in the form of "rewards".
In addition to measuring the performance of an agent, rewards can be used as a learning metric for an intelligent Controller (\todo{ref controller schema}).

\subsubsection{Short Term Rewards}
Short term rewards are given each time an agent performs an action to reflect the immediate outcome of the action.
Energy and health depletion are major factors in this reward.
If the action required an excessive amount of energy or resulted in damage to the agent, the reward is decreased.
Other factors that come into play depend on the specific action taken.
If the agent attempted to move to a new area and failed to move (ex: a hill was too steep to climb) a deduction is made.
A small increase in reward is applied if the agent moves into an unexplored area.
If the agent uses a scanner, the reward is adjusted to reflect the amount of new data learned.
This penalizes the agent from using a scanner twice in a row or after small movements as it is not efficient use of energy.
\todo{Add STR Equation}

\subsubsection{Long Term Rewards}
Long term rewards are calculated once the operation is over.
Operation end cases are when the agent has successfully completed its goal, or it is depleted of health or energy.
To reflect these scenarios, the reward is determined by the goal completion, remaining health and remaining energy.
Even if a goal is completed, the agent could receive a low score if it was "reckless" and took lots of damage or used large amounts of energy.
The long term reward is then propagated backwards through all the actions that were taken.
The actions performed immediately before the end of the operation are given highest score.
Previous actions then receive diminishing reward based on \todo{LTS reward distribution equation}.
\todo{Add LTR Equation}



% ==============================================================================
% ENVIRONMENT GENERATION
% ==============================================================================
\section{Environment Builder}
The SCOUt EnvironmentBuilder is a tool for creating diverse Environment models, while remaining simple to implement and understand.
The tool is highly abstracted so that more details can easily be added to the model as needed while still maintaining a defined build process.
Environments are procedurally generated based upon a collection of parameters called an environment template.
These templates require minimal input to build a dynamic range of possible Environments.
An Environment can be tweaked or even built entirely by hand, but the procedural generation process removes this overhead.

Procedural Generation Process
\begin{enumerate}
  \item Environment Template is passed in
  \item	Builder initializes a grid of empty Cells
  \item	ElementSeeds are used to populate each present element type into the gird of cells
  \item	TerrainModifications are applied to manipulate their related element(s)
  \item	Anomalies are placed randomly within the environment
  \item	Anomaly effect(s) are applied to corresponding element(s) in neighboring Cells
\end{enumerate}


\subsection{Environment Templates}
Each template created will act as a guide in the creation of an instance of the Environment class.
A template can create similar, but unique environments each time it is used by the EnvironmentBuilder.
This allows testing and training agent controllers multiple times in similar conditions, while still providing a dynamic range of scenarios that the Agent may face in each generated Environment.
Each template is comprised of the name, dimensions and scale of the environment along with lists of ElementSeeds, TerrainModifications and Anomalies to be applied.

\begin{lstlisting}[language=Scala]
class EnvironmentTemplate (
  name: String
  height: Int
  width: Int
  scale: Double
  elementSeeds: List[ElementSeed]
  terrainModification: List[TerrainModification]
  anomalies: List[Anomaly]
)
\end{lstlisting}


\subsection{Element Seeds}
The environment builder begins by procedurally generating one Layer of Elements at a time.
Each Element class has a companion class called an ElementSeed which holds parameters used to produce a Layer of its element type, and a unique function defining how procedural generation will take place to produce the Layer.
The generation of each Layer is modeled on how the element type's values may vary in a real-world scenario.
Parameters within each Seed are set to default values that can also be overriden by creating a new instance of the ElementSeed.
This can change how the values within the Layer will vary.

\begin{lstlisting}[language=Scala]
trait ElementSeed (
  elementType: String
  function buildLayer(height, width, scale)
)
\end{lstlisting}

The environment builder will use each ElementSeed to produce each Layer of Elements.
Resulting Layers will be temporarily stored in a list until the end of the build process so they can easily be manipulated before being stored into corresponding Cells within the Environment grid.
Some layers are far easier to generate than others.
Latitude and Longitude layers can simply be generated by calculating the distance each cell is from the origin point on the Environment grid (cell (0,0)).

For an example, let’s look at the ElementSeed for producing the Elevation Layer.

\begin{lstlisting}[language=Scala]
Class ElevationSeed (
  elementType: String = "Elevation"
  average: Double = 0.0
  deviation: Double = 1.0
) {
  function buildLayer(height, width, scale) {
    val layer: Layer = new Layer(AB.fill(height)(AB.fill(width)(None)))
    for {
      x <- 0 until height
      y <- 0 until width
    } {
      val value = randomDeviation(average)
      layer.setElement(x, y, new Elevation(value))
    }
    layer.smoothLayer(3, 3)
    return layer
  }

  function randomDeviation(average) {
    val lowerBound = average - deviation
    val upperBound = average + deviation
    return randomDouble(lowerBound, upperBound)
  }
}
\end{lstlisting}


The ElevationSeed’s buildLayer algorithm first initializes an empty Layer.
Next, it sets each (x,y) coordinate in the Layer to an a random Elevation value within a standard deviation of the average value provided [(average - deviation), (average + deviation)].
Once every Cartesian position has been set to an instance of Elevation, the layer is then smoothed.
Smoothing is a function defined within the Layer class that will reduce strong variations of Element values within the Layer.
For Elevation, this would equate to transforming a highly rigid surface into a smoother, more natural surface.

\todo{smoothing pseudo-code}
\todo{smoothing before-after photo?}


\subsection{Terrain Modifications}
Terrain modifications influence the basic landscape of the environment.
After each ElementSeed has produced a layer for the Environment, TerrainModifications are applied one after the other, taking care not to overlap modifications (for example, we wouldn't want a hill to overlap with a valley and cancel each other out).
Each modification represents a severe alteration of one or more element type Layers within the environment.
Following a similar process laid out by Doran and Parberry\cite{doran_controlled_2010}, desired alterations are incorporated into the environment while allowing unique variations of each to develop.
Their controlled procedural generation process is used to produce landmasses that potentially have bodies and channels of water.
SCOUt's environment builder generalizes this process and extends it to allow multitudes of element types to be modified.
The TerrainModification trait provides an extendable template from which all types of modifications that can be applied.

\begin{lstlisting}[language=Scala]
trait TerrainModification (
  name: String
  elementTypes: List[String]
) {
  function modify(layers: List[Layer])
}
\end{lstlisting}


For an example, lets look at Elevation again.

\begin{lstlisting}[language=Scala]
class ElevationModification (
  name: String = "Elevation Modification"
  elementType: List[String] = List("Elevation")
  modification: Double
  deviation: Double
  coverage: Double
  slope: Double
) {
  def modify(layer: Layer, constructionLayer: ConstructionLayer) = constructionLayer.getRandomUnmodified() match {
    case None => // No unmodified cells
    case Some(startCell) => {
      // Set local variables
      var modifiedCells: AB[(Int,Int)] = AB()
      val numCellsToMod = Math.round(coverage * constructionLayer.cellCount).toInt
      // Initial modification
      val startX = startCell._1
      val startY = startCell._2
      layer.setElementValue(startX, startY, modification)
      constructionLayer.setToModified(startX, startY, "elevation")
      modifiedCells.append((startX, startY))
      // Move to random, unmodified neighbors and modify
      for (i <- 0 until numCellsToMod) constructionLayer.getNextUnmodifiedNeighbor(modifiedCells) match {
        case None => // No neighbor cells to modify
        case Some((x,y)) => {
          val currentValue = layer.getElementValue(x, y).getOrElse(0.0)
          val mod = randomDouble((modification - deviation), (modification + deviation))
          val newValue = currentValue + mod
          layer.setElementValue(x, y, newValue)
          constructionLayer.setToModified(x, y, "elevation")
          modifiedCells.append((x, y))
        }
      }
      // Apply sloping factor to modified area through smoothing
      val effectedRadius = Math.abs(Math.round(modification / slope).toInt)
      for (i <- 0 until modifiedCells.length) {
        val randomIndex = randomInt(0, modifiedCells.length - 1)
        val c = modifiedCells.remove(randomIndex)
        val originX = c._1
        val originY = c._2
        for {
          x <- (originX - effectedRadius) to (originX + effectedRadius)
          y <- (originY - effectedRadius) to (originY + effectedRadius)
          if dist(x, y, originX, originY) != 0
          if dist(x, y, originX, originY) <= effectedRadius
        } layer.smooth(x, y, 2, dist(originX, originY, x, y))
      }
    }
  }
}
\end{lstlisting}
\todo{turn into pseudo-code}

Here we have an ElevationModification which will allow us to create hills and valleys within the environment.
Again following the approach of \cite{doran_controlled_2010}, random, unmodified (x,y) positions are selected from the Layer to begin with and updates their value to the specified modification value.
The modifier then performs "walks" to random, unmodified neighboring Elements, updating their values within a standard deviation of the specified modification value.
These walks continue until the specified coverage area has been modified, or until there are no neighboring cells that can be modified.
A special Layer smoothing algorithm is then applied to the Elevation values in the modified area, as well as the immediate surrounding unmodified area to reduce rigidity and give a more natural change in values between neighboring cells.
This type of smoothing applies a given slopping factor within the modified area, allowing the ElevationModification to generate gentle hills or valleys or sharp cliffs depending on the slope defined.


\subsection{Anomaly Placement}
Once all TerrainModifications have been applied, anomalies are placed into the environment.
Each specified anomaly is randomly placed into cell(s) in the Environment.
For anomalies that occupy more than one cell, neighboring cells are chosen at random until the Anomaly's coverage area is met, or there are no neighboring cells that can contain the Anomaly.
The anomaly type is appended to each occupied Cell's anomalies list for reference within the simulation.
After an anomaly has been placed, each of the Anomaly's Effects are applied.
An Effect will alter the Element values for the occupied and surrounding Cells in the effected area.
These alterations are typically applied as a "radiation".
For example, a Human Anomaly might radiate heat and sound.
To account for this, the radiation function is applied to the temperature and decibel values of cells in the effected radius.
\todo{radiation function}

Now that all ElementSeeds, TerrainModifications and Anomaly placements have occured, the resulting Layers containing thier respective Elementss are populated into their corresponding (x,y) Cell location within the Environment grid.
The resulting instance of an Environment class is then returned by the builder to the requesting party.



% ==============================================================================
% VISUALIZATION TOOL
% ==============================================================================
\section{Visualization Tool}
The environment build tool provides a Graphical User Interface (GUI) for creating and visualizing environments.
Electron \todo{Possibly cite Electron} is used to simulate a web page contained within a standalone desktop application.
This allows the front end to be written in JavaScript, HTML and CSS and handle communication to the back end via Http over a localhost network.
Scala library HTTP4s \todo{Possibly cite http4s} is used to create a server on a localhost network for handling the http requests from the front end.
Launching the GUI starts up the Scala server in a new terminal and opens the Electron window which will begin attempts to establish communication with the server.


\subsection{Home Page}
\todo{add photos}
Once connection between the server and GUI has been established, the user is brought to the Home Page wher they can choose to generate a random environment, build a custom environment, load in an environment or view an operation.
For a random environment, the user only inputs the name and n x m size of the environment and all other variables are selected by the server.
Building a custom environment steps the user through a series of form pages to create an EnvironmentTemplate.
Loading an environment allows the user to select a saved environment or a saved template to use.
Selecting an operation will load the environment and log of all actions taken by an agent during a specific operation run that is saved in memory.
Once an environment has been generated and/or loaded by the server, it is returned to the GUI to be displayed.
The environment build tool will parse the returned environment into a graphical data representation, with interactive capabilities to explore the specific variables within the environment.
In the case that the user selected an operation, the user will additionally be able to step through the event log of an Operation.


\subsection{Template Forms Page}
\todo{add photos}
To create a template, the user will be presented with a series of forms with parameter input fields.
The forms are generated based on the available Element, TerrainModification and Anomaly classes that are defined in SCOUt's back end.
The first three forms will ask the user which element types, terrain modification and anomaly types they would like to include.
Some elements (environment, latitude and longitude) are required in all environments.
As the user selects what will be present in the environment, more forms will be generated for them to provide seed data (for Elements and Anomalies), or parameter definitions (for TerrainModifications).
Each form within this process will save the user’s input data on the front end.
This allows the user to go back and edit form data while moving through the different form pages, as well as return and edit the form data after an Environment has already been generated and loaded into the \ref{Visualizer Page}.
Form data is also set within required bounds and checked before submission.
Once the user has filled out all required form info, they can review their entire form entry from a single page and then submit it.
When submitted, the front end data is converted into Json data and sent in a request to the SCOUt server via a JavaScript fetch request.
An Environment instance is then built on the back end using the template parameters provided, and returned to the front end where it is loaded into the \ref{Visualizer Page}.


Template Form Sections
\begin{enumerate}
  \item Environment Name and Size
  \item Element Types Present
  \item Terrain Modifications Present
  \item Anomalies Present
  \item Element Seeds
  \item Terrain Modification Templates
  \item Anomaly Seeds
\end{enumerate}


\subsection{Visualization Page}
\todo{add photos}
The visualization page provides an interactive overview of any given environment.
The main focus is on the display section where the entire environment grid is represented using heatmaps.
Different Element Layers can be viewed independently, anomaly locations can be highlighted, and specific element type values of a single cell can be viewed.
A main menu is also present to allow a user to perform higher level actions.
All of these interactive features are controlled by action, toggle and radio buttons within different sections of the visualization page.
The primary use of the visualization page is for creating environment templates and for debugging.
Debugging usage ranges from analyzing an environment that was used for testing an agent-controller setup or new features.
Examples of new features would be adding new classes (ex: a new Element type), or altering the process in which environments are generated and stored on the back end.


\subsubsection{Main Menu}
\todo{add photos}
The main menu provides high level functions to perform while using the environment build tool.
The main menu options are displayed at the top of the visualizer as a series of buttons.
The user can select buttons to return to the home page, regenerate or save the current EnvironmentTemplate (if an environment template is being used), or save the current Environment instance that is currently being viewed.
If the user wants to tweak the current template that is in use, they can do so by returning to the Home Page\ref{Home Page} via the home button, and choosing "Custom Environment" again.
Their previously set parameters will be loaded back into the form fields for editing.


\subsubsection{Display}
\todo{add photos}
The display is laid out in a grid of display cells corresponding to the Environment's Cell grid.
A display layer is created for each element type present within the environment using the D3 library.
D3's heatmap creates a graphical representation of data values over a 2-dimentional space, providing a solution for visually differentiating each cell's value within the environment cell grid.
Heatmaps display the variation of values using a color scale where higher values are indicated by darker sections and lower values as a lighter section of the map.
For example, when viewing the elevation's heatmap, a hill will appear darker than a valley.
The user also has the ability to select individual cells by clicking on their region in the displayed cell grid.
The display will highlight a display cell once it has been selected and then load the cells data into the Legend\ref{Legend}.


\subsubsection{Tool Bar}
\todo{add photos}
The tool bar is divided into three subsections: Toggle Layers, Current Layer and Current Anomaly.
The Toggle Layers subsection provides two toggle buttons for the user to turn on and off the display of the Elevation layer and the Grid layer.
The Elevation layer is a grey-scale contour map of the Elevation layer created by D3 heatmap (a contour map is the same as a heatmap, with the distinction of boarder lines between each value layer).
Because Elevation is the most fundamental piece to any environment, it is the only element type whose layer has the option of always being displayed.
The grid layer displays solid black lines between each cell within the cell grid.
The Current Layer subsection provides a list of radio buttons for all other element types present within the environment.
When one of these radio buttons is selected, a green-scale, transparent heatmap of the selected element type will be populated into the display.
This element type layer will be displayed on top of the Elevation layer (if Elevation is toggled on).
Only one element type layer can be viewed at a time to prevent crowding the display.
The Current Anomaly subsection is also a set of radio buttons for each anomaly type present in the environment.
Selecting one of these will highlight all display cells containing the given anomaly type in red.
Just as the case with element type layers, only one anomaly type can be viewed at a time.


\subsubsection{Legend}
\todo{add photos}
The legend provides an overview of the environment in three main subsection: environment, layer and cell.
For the environment subsection, the name of the current environment is displayed along with the dimensions and minimum and maximum elevation within the environment.
The layer subsection displays the minimum and maximum values of the selected display layer (if one is selected), as well as the display layer's value at the selected cell (if a cell is selected).
When a cell is selected, the values of all element types in the area covered by the cell are presented in a list, as well as the cell's relative coordinates in the grid.


\subsubsection{Operation Log}
\todo{add photos}
The operation log section is a special section that only appears in the visualizer when the user loads an Operation run.
This section has buttons that allow the user to step through each event that took place during the given agent's operation.
The user can select to step forward or backwards by 1 or 10 events.
When each event is loaded into the visualizer, the display section will update by selecting the cell where the agent is currently located.
There is also a text display section that shows the index of the event that is currently being viewed, the action that was chosen, the health and energy of the agent during this event, and the long and short term rewards that were received.


% GOOD
\chapter{CONTROLLERS} \label{ch:controllers}
Three types of control schemas are compared in this project: random, heuristic, and SCOUt's memory-based learning.
All three controllers are designed to operate within unknown environments using whatever sensors are available.
Controllers are compared based on their ability to complete a defined goal, the number of actions that the controller had to perform before completing the goal, and the remaining health and energy levels of the agent.
The random controller will select valid actions at random until the goal is completed successfully, or the agent's health or energy is depleted.
This behavior provides a primary baseline for determining what levels of performance are considered \textit{intelligent}.
Intelligent controllers would need to exceed the performance of a controller that simply selects actions at random.
Both the memory-based learning and heuristic approaches can be considered intelligent, as they use knowledge of their environment to select actions.
It is up to each controller type to effectively use the information provided in the agent's current state to guide them towards success.
Heuristic controllers perform a set of hard-coded logical analyses to choose actions.
This type of approach offers practical solutions to operations but are not expected to be optimal.
In addition to this, heuristic controllers are not adaptive to new situations as their logical schemas must be defined for each specific goal.
Experiments in chapter~\ref{ch:experiments_and_results} simulate operations for two goals: \textit{Find Human} and \textit{Map Water}.
Separate heuristic controllers are created for each of these and provide a secondary performance baseline.
For SCOUt's memory-based learning schema to be considered both intelligent \textit{and} adaptive, it would need to perform at the same level or better than the heuristic schemas designed specifically for each goal.
The architecture of the heuristic and SCOUt control schemas are covered in section~\ref{sec:heuristic_controllers}~and~\ref{sec:scout_controller} respectively.


\section{Heuristic Controllers} \label{sec:heuristic_controllers}
Two heuristic controllers are used in testing: $Heuristic_{FH}$ and $Heuristic_{MW}$.

\noindent
$Heuristic_{FH}$ is designed for the \textit{Find Human} goal, and $Heuristic_{MW}$ is designed for \textit{Map Water}.
Both use the same action decision models (figure~\ref{fig:heuristic_decision_model}) with slight variations.
The models will consider every valid action and give each a score based on the agent's current state.
The action with the highest score is then selected.
Different score calculations are used for scanning and movement actions, but scores will always be a value between 0 and 1 (1 being the best possible score).
The difference between the two heuristic controllers is found in the way they score movement actions.
$Heuristic_{FH}$ influences movement to cells that have higher decibel and temperature differentials, as a human anomaly will likely be indicated by increased values of these element types.
$Heuristic_{MW}$ encourages movement into quadrants that have fewer known element values so that it can gather new data from unexplored area.
Both controllers' movement-action scores also factor in hazard avoidance.
Movement into cells with the presence of water or large elevation differentials is discouraged as they could result in damage to the agent.
During an operation, the heuristic controller keeps a history of actions performed at each $(x,y)$ location in the environment.
After action scores are initially calculated using their respective function, a penalty will be given to any repetitive actions.
If the controller has previously selected one of the considered actions while in the same location, the calculated score will be cut in half.
This will encourage the controllers to make new choices resulting in exploration of new areas, and a more efficient use of sensors.

\begin{figure}[H]
  \centering
  \includegraphics[width=1.0\columnwidth]{Figures/heuristic_decision_model.png}
  \caption{The general decision model that our heuristic controllers follow. An \texttt{AgentState} and a list of valid actions are passed to the controller. The controller then assigns a score to each action by analyzing related attributes within the \texttt{AgentState}. The highest scoring action is then returned.}
  \label{fig:heuristic_decision_model}
\end{figure}

Valid scanning actions are all scored using function~\ref{algorithmic:score_scan_action}.
Higher scores will be given to scanning actions for an element type that is considered more important and has fewer known values within the corresponding sensor's range.
Importance of an element type is determined by whether it is flagged as hazardous and/or as an indicator.
The amount of known values in the corresponding sensor's range is calculated by referencing the agent's \texttt{internalMap}.
The resulting score should influence the controller to use sensors efficiently, assist with hazard avoidance, and emphasize goal completion.

\begin{algorithm}[H]
  \setstretch{1.35}
  \caption{Calculate a score for a considered scanning action for a specific element type based on an \texttt{ElementState}. The returned result will be used to rank the action in the decision-making process. $W_{item}$ denotes the attributed weight for $itemReward$.}
  \begin{algorithmic} \label{algorithmic:score_scan_action}
    \REQUIRE $W_{indicator} \in \left[0, \infty \right)$
    \REQUIRE $W_{hazard} \in \left[0, \infty \right)$
    \REQUIRE $W_{pkir} \in \left[0, \infty \right)$
    \REQUIRE $W_{immediates} \in \left[0, \infty \right)$
    \REQUIRE $indicator \in \{true, false\}$
    \REQUIRE $hazard \in \{true, false\}$
    \REQUIRE $percentKnownInRange \in \left[0, 1 \right]$
    \REQUIRE $immediatesKnown \in \left[0, 4 \right]$
    \ENSURE $scanActionScore \in \left[0, 1 \right]$
    \IF {indicator = true}
      \STATE $iScore \leftarrow W_{indicator}$
    \ELSE
      \STATE $iScore \leftarrow 0$
    \ENDIF
    \IF {hazard = true}
      \STATE $hScore \leftarrow W_{hazard}$
    \ELSE
      \STATE $hScore \leftarrow 0$
    \ENDIF
    \STATE $pkirScore \leftarrow (1 - percentKnownInRange) \times W_{pkir}$
    \STATE $imdsScore \leftarrow ((4 - immediatesKnown) / 4) * W_{immediates}$
    \STATE $scoresTotal \leftarrow iScore + hScore + pkirScore + imdsScore$
    \STATE $W_{total} \leftarrow W_{indicator} + W_{hazard} + W_{pkir} + W_{immediates}$
    \RETURN $scanActionScore \leftarrow scoresTotal / W_{total}$
  \end{algorithmic}
\end{algorithm}


Scoring each valid movement actions is based on the controller's specific implementation of the \texttt{scoreMovmentAction} function.
These functions involve a series of sub-functions tied to each available sensor's element type.
Each of the sub-functions calculate a sub-score for their element type.
These sub-functions use threshold analyses on the \texttt{QuadrantState}s corresponding to the direction of movement being considered.
Once each element type's sub-score has been returned to the \texttt{scoreMovmentAction} function, an overall score is determined by a weighted average.
The overall scoring functions used for $Heuristic_{FH}$ (algorithm~\ref{algorithmic:findHuman_scoreMovmentAction}) and $Heuristic_{MW}$ (algorithm~\ref{algorithmic:mapWater_scoreMovementAction}) follow the same logic, but contain different sub-functions and related weights.
For an example of how threshold analyses are conducted within a sub-function, see $Heuristic_{FH}$'s \texttt{scoreElevation} algorithm (algorithm~\ref{algorithmic:findHuman_scoreElevation}).


\begin{algorithm}[H]
  \setstretch{1.35}
  \caption{Calculate a score for a considered movement action in a specific direction based on a set of corresponding \texttt{QuadrantState}s ($QS$). The returned results will be used to rank the action in the decision-making process. $W_{item}$ denotes the attributed weight for $itemReward$. This function also uses a $score<Element-Type>$ function. Example for one such equation is algorithm~\ref{algorithmic:findHuman_scoreElevation}. This equation is used specifically for the $Heuristic_{FH}$ controller's decision model.}
  \begin{algorithmic} \label{algorithmic:findHuman_scoreMovmentAction}
    \REQUIRE $W_{elevation} \in \left[0, \infty \right)$
    \REQUIRE $W_{decibel} \in \left[0, \infty \right)$
    \REQUIRE $W_{temperature} \in \left[0, \infty \right)$
    \REQUIRE $W_{water} \in \left[0, \infty \right)$
    \ENSURE $scoreElevation \rightarrow \left[0, 1 \right]$
    \ENSURE $scoreDecibel \rightarrow \left[0, 1 \right]$
    \ENSURE $scoreTemperature \rightarrow \left[0, 1 \right]$
    \ENSURE $scoreWater \rightarrow \left[0, 1 \right]$
    \ENSURE $movementActionScore \in \left[0, 1 \right]$
    \STATE $eScore \leftarrow scoreElevation(QS) * W_{elevation}$
    \STATE $dScore \leftarrow scoreDecibel(QS) * W_{decibel}$
    \STATE $tScore \leftarrow scoreTemperature(QS) * W_{temperature}$
    \STATE $wScore \leftarrow scoreWater(QS) * W_{water}$
    \STATE $scoresTotal \leftarrow eScore + dScore + tScore + wScore$
    \STATE $W_{total} \leftarrow W_{elevation} + W_{decibel} + W_{temperature} + W_{water}$
    \RETURN $movementActionScore \leftarrow scoresTotal / W_{total}$
  \end{algorithmic}
\end{algorithm}


\begin{algorithm}[H]
  \setstretch{1.35}
  \caption{Calculate a score for a considered movement action in a specific direction based on a set of corresponding \texttt{QuadrantState}s ($QS$). The returned results will be used to rank the action in the decision-making process. $W_{item}$ denotes the attributed weight for $itemReward$. This function also uses a $score<Element-Type>$ function. Example for one such equation is algorithm~\ref{algorithmic:findHuman_scoreElevation}. This equation is used specifically for the $Heuristic_{MW}$ controller's decision model.}
  \begin{algorithmic} \label{algorithmic:mapWater_scoreMovementAction}
    \REQUIRE $W_{elevation} \in \left[0, \infty \right)$
    \REQUIRE $W_{water} \in \left[0, \infty \right)$
    \ENSURE $scoreElevation \rightarrow \left[0, 1 \right]$
    \ENSURE $scoreWater \rightarrow \left[0, 1 \right]$
    \ENSURE $movementActionScore \in \left[0, 1 \right]$
    \STATE $eScore \leftarrow scoreElevation(QS) * W_{elevation}$
    \STATE $wScore \leftarrow scoreWater(QS) * W_{water}$
    \STATE $scoresTotal \leftarrow eScore + wScore$
    \STATE $W_{total} \leftarrow W_{elevation} + W_{water}$
    \RETURN $movementActionScore \leftarrow scoresTotal / W_{total}$
  \end{algorithmic}
\end{algorithm}


\begin{algorithm}[H]
  \setstretch{1.35}
  \caption{Calculate a score for a \texttt{QuadrantState} ($Q$) of element type ``elevation.'' The returned results will be used to rank the action in the decision-making process. $W_{item}$ denotes the attributed weight for $itemReward$. This equation is used in both the $Heuristic_{FH}$ and $Heuristic_{FH}$ controllers' decision models.}
  \begin{algorithmic} \label{algorithmic:findHuman_scoreElevation}
    \REQUIRE $W_{percentKnown} \in \left[0, \infty \right)$
    \REQUIRE $W_{immediateValue} \in \left[0, \infty \right)$
    \ENSURE $percentKnown \rightarrow \left[0, 1 \right]$
    \ENSURE $scanMovementScore \in \left[0, 1 \right]$
    \STATE $pkScore \leftarrow scoreElevation(QS) * W_{elevation}$
    \IF {$\exists immediateValue$}
      \IF {$|immediateValue| > 12$}
        \STATE $imScore \leftarrow 1$
      \ELSE
        \STATE $imScore \leftarrow 0$
      \ENDIF
    \ELSE
      \STATE $imScore \leftarrow 0$
    \ENDIF
    \STATE $scoresTotal \leftarrow pkScore + imScore$
    \STATE $W_{total} \leftarrow W_{elevation} + W_{decibel} + W_{temperature} + W_{water}$
    \RETURN $scanActionScore \leftarrow scoresTotal / W_{total}$
  \end{algorithmic}
\end{algorithm}


\section{SCOUt Controller} \label{sec:scout_controller}
The SCOUt controller uses reinforcement learning to build a memory of past actions and rewards for planning future actions.
After each operation, the SCOUt controller will store state-action rewards in memory.
A state-action reward (SAR) contains the action that the agent took, the state that the agent was in when it chose this action, and the short-term and long-term rewards that the agent received.
In future operations, the SCOUt controller (figure~\ref{fig:scout_decision_model}) will search in memory to find SARs with states that are similar to the agent's current state.
Utilizing data from the agent's current state and the controller's collection of past action-state pairs, SCOUt will predict rewards for each possible action and select one based on these predictions.

\begin{figure}[H]
  \includegraphics[width=1.0\columnwidth]{Figures/scout_decision_model.png}
  \caption{Action decision model for the SCOUt controller. The controller searches for memory state-action rewards (SAR) that have a similar state to the current state. The chosen actions and rewards for each similar SAR is used to create a score for each valid action that the can be taken in the current state.}
  \label{fig:scout_decision_model}
\end{figure}

Calculations used for action decision rely on several weights and variables to assist in state comparisons and future reward prediction.
Because of the large number of weights required for these calculations, a basic genetic algorithm (GA) was used to optimize these weights.
The GA initialized a population of 10 weight sets and evolved them for 50 generations.
Each generation creates five mutated copies and five crossover copies of individuals in the current population.
The individuals that are copied for mutation or crossover are chosen using roulette selection.
% \todo{ref roulette selection algo}
% \todo{ref GA algo}
Fitness scores are calculated for each of the resulting 20 individuals based on their performance within a series of 50 operations.
Ten survivors are then selected for the next generation.
Survivor selection keeps the two individuals with the highest fitness scores and uses roulette selection for choosing the remaining seven.
The weight set with the highest fitness in the final generation was selected for use in experimentation, and its values listed in table ~\ref{table:evolved_weight_set}.

% \todo{sizing}
\begin{table}
  \centering
  % \small
  \def\arraystretch{1.25}
  \caption{Set of variables and weights used by the SCOUt controller for action decision. These variables/weights were produced using a basic genetic algorithm.}
  \resizebox{\textwidth}{!}{%
  \begin{tabular}{ l l c }
    \textbf{Attribute}                & \textbf{Description}            & \textbf{Variable/Weight} \\
    \hline
                                      & \textbf{State Comparison Weights}& \\
    health                            & remaining health               & 0.41  \\
    energy                            & energy level                & 0.78  \\
    elementStates                     & overall element states                & 0.61  \\
    quadrantStates                    & overall quadrant states                & 0.16  \\
    \hline
                                      & \textbf{Element State Comparison Weights}& \\
    indicator                         & element type is indicator                & 0.31  \\
    hazard                            & element type is hazardous                & 0.07  \\
    percentKnownInRange               & known element type values in range of sensor                & 1.0   \\
    immediateKnown                    & number of immediate cell values known                & 0.41  \\
    \hline
                                      & \textbf{Quadrant State Comparison Weights}& \\
    indicator                         & element type is indicator                & 0.38  \\
    hazard                            & element type is hazardous                & 0.23  \\
    percentKnown                      & known element type values in quadrant                & 0.2   \\
    averageValue                      & average element value in quadrant cells                & 0.19  \\
    immediateValue                    & immediate quadrant cell value                & 0.29  \\
    \hline
                                      & \textbf{Action Selection}& \\
    similarityThreshold               & SAR comparison qualification  & 0.26  \\
    minimumSimilarStates              & used to calculate prediction ``confidence''    & 10    \\
    repetitionPenalty                 & penalty for action that would be repetitive                & 0.1   \\
    \hline
                                      & \textbf{Movement Action Score Weights}& \\
    predictedShortTermReward          & action's predicted short-term reward       & 0.87  \\
    predictedLongTermReward           & action's predicted long-term reward      & 0.45  \\
    confidence                        & confidence in predicted rewards       & 0.25  \\
    \hline
                                      & \textbf{Scanning Action Score Weights}& \\
    predictedShortTermReward          & action's predicted short-term reward       & 0.61  \\
    predictedLongTermReward           & action's predicted long-term reward      & 0.34  \\
    confidence                        & confidence in predicted rewards       & 1.0   \\
    \hline
  \end{tabular}}
  % \caption{Table ~\ref{table:evolved_weight_set}: Evolved Weight Set}
  \label{table:evolved_weight_set}
\end{table}


\subsection{Memory}
The SCOUt controller can gather memory from every operation.
When an operation has finished, and long-term rewards have been assigned to each action, the controller creates new SARs, and selects a sub-set of them to be stored in memory.
The rest are discarded.
Saving only a sub-set cuts back on the size of the memory file as well as the computational time that will be required to search for similar states.
The current memory selection method in this project's implementation saves the last 20 SARs, and a uniformly sampled sub-set of the remaining SARs from the operation.
The last 20 are always saved because they typically hold the most important events leading up to the success or failure of the operation.
To also capture events that occur during the agent's initial and intermediate search process, the controller retains 5 percent of the remaining SARs generated.
These SARs are uniformly sampled beginning at index 0.
So, if there were 100 SARs in the remaining set, SARs indexed 0, 20, 40, 60, and 80 would be stored in memory.
Each SAR is added to the controller's memory file as a JSON object.
The next time that the controller is used in operation, the collection of SARs can then be decoded from the file back into \texttt{StateActionReward} class instances.


\subsection{State Normalization}
The data within each SAR's state is relative to the operation in which they were recorded.
To handle variances found between all of the states stored in memory, SCOUt normalizes them using a Gaussian approach as suggested by McCaffrey~\cite{mccaffrey_how_nodate}.
Normalization helps make data values more meaningful when studied by the controller.
For example, if the controller was seeking out a human, it may look for increases in decibel values.
In order for the controller to determine how much of an increase is significant enough to investigate, it needs to first understand what variations are considered normal.
Gaussian distribution provides this functionality through the calculation of mean and standard deviation ($sd$) values in a data set.
If the agent has gathered decibel readings in its north quadrant that are well outside the $sd$ found in the controller's memory, it should be encouraged to investigate.
All numerical attributes within an \texttt{AgentState} are normalized using this Gaussian method.
This applies to health and \texttt{energyLevel} within \texttt{AgentState}s, \texttt{percentKnownInSensorRange} within \texttt{ElementState}s, and \texttt{percentKnown}, \texttt{averageValueDifferential}, and

\noindent
\texttt{immediateValueDifferential} within \texttt{QuadrantState}s.

The normalization process begins by extracting each of these attributes from all SAR states within the loaded memory.
Next the mean and standard deviation values are calculated and stored in an instance of a \texttt{GuassianData} class (Appendix~\ref{appendix:gaussian_data}).
Once mean and standard deviation values are known, the controller will go back through every SAR's state and normalize their attributes using each corresponding \texttt{GaussianData} class instance.
The normalization function (equation~\ref{eq:gaussian_normalize}) will produce a ``normal'' value that reflects how many standard deviations the attribute falls above or below the mean.
A value of 0 represents no difference between the attribute's value and the mean, values of 1 and -1 represent a difference of one standard deviation from the mean, and so on.
When SCOUt searches for similar states, it will also normalize the current state using the existing \texttt{GuassianData} instances.
By normalizing the current state against the states in memory, the numerical attributes compared will all be relative to the mean of the values held in memory.

\begin{capeq}
  \begin{equation} \label{eq:gaussian_normalize}
    x_{normal} = \frac{(x - m)}{sd}
  \end{equation}
  \caption{Normalization of an attribute value, $x$, based on the gaussian mean, $m$, and gaussian standard deviation, $sd$, for the given attribute.}
\end{capeq}


% \todo{major revisions to this section}
% Replace as much as possible with graphics
% \todo{eliminate weighted average refs?}

\subsection{State Comparisons} \label{subsec:state_comparisons}
Now that all state attributes are normalized, the controller can use a more intuitive approach for calculating the differences between two states.
State comparisons are used to build a set of SARs from memory that contain states similar to an agent's current state.
These SARs will later be used to assist in reward prediction.
For an SAR to qualify for addition into this set, its state must have an overall difference below the \textit{similarityThreshold} specified in table ~\ref{table:evolved_weight_set}.
Overall state difference is calculated using a series of difference calculations between related attributes in the two compared \texttt{AgentState}s.
Results from the series of difference calculations will all be collapsed into a single \textit{overallStateDifference} using a weighted average function (equation ~\ref{eq:weighted_average}).
By comparing each attribute separately and applying a weighted average to the resulting difference calculations, this allows the controller to assign a level of importance to each individual attribute.
Importance is assigned via weight values that are between 0 and 1 (see ``state comparison'' weights in table~\ref{table:evolved_weight_set}).
The higher the attribute's weight it, the more influence it will have in the overall state difference.
An attribute with a weight of 0 will be completely ignored in a weighted average equation.
Different weighted average equations are used for \textit{overallStateDifference} calculation depending on whether the considered SAR's action is a movement or scanning action.
This allows the controller to compare only the attributes that are relevant to the type of action that was selected.

\begin{capeq}
  \begin{equation} \label{eq:weighted_average}
    WeightedAverage = \frac{\sum_{i=0}^{n} A_{i} * W_{i}}{\sum_{i=0}^{n} W_{i}}
  \end{equation}
  \caption{A general equation that takes a list of $n$ attribute values ($V$) and a list of $n$ corresponding weights ($W$) and calculates a weighted average of all attribute values.}
\end{capeq}

Difference comparisons for each attribute in an \texttt{AgentState} are calculated based on their data type (boolean, normalized numerical value, optional normalized numerical value, or sub-class).
Sub-class comparisons, such as comparing two \texttt{ElementState}s, follow the same procedure as \texttt{AgentState} does for calculating an overall difference.
Difference comparisons will be made for each of the attributes within the sub-class, and a weighted average function is applied to the results.
Boolean differences will return 0 when the compared attributes are both true or both false and return 1 otherwise (equation ~\ref{eq:boolean_difference}).
For example, \textit{BooleanDifference} is used to calculate whether an element type in two \texttt{ElementState}s were both flagged as an indicator or not.
Normalized numerical attributes follow the \textit{GaussianDifference} equation (equation ~\ref{eq:gaussian_difference}).
This equation will produce values that hold the same principle as the normalization process, where the closer the difference is to 0, the more similar they are.
If two values are identical, their \textit{GaussianDifference} will be 0.
Otherwise, the \textit{GaussianDifference} will be relative to how many standard deviations away from each other the two values are.
% Proofs for these behaviors are found in appendix item~\ref{appendix:gaussian_difference_identical} and appendix item~\ref{appendix:gaussian_difference_different} respectively.
Optional values follow a unique case-based equation (equation~\ref{eq:option_difference}) to calculate the \textit{GuassianDifference} only when both are known.
If one value is known and the other is not, a difference of 1 is returned.
If both values are unknown a difference of 0 is returned.

\begin{capeq}
\begin{equation} \label{eq:boolean_difference}
  BooleanDifference = \begin{cases}
    x = y & 0 \\
    x \neq y & 1
  \end{cases}
\end{equation}
\caption{Difference calculation for two boolean values, $x$ and $y$.}
\end{capeq}

\begin{capeq}
\begin{equation} \label{eq:gaussian_difference}
  GaussianDifference = |x_{normal} - y_{normal}|
\end{equation}
\caption{Difference calculation for two normalized vales, $x$ and $y$.}
\end{capeq}

\begin{capeq}
\begin{equation} \label{eq:option_difference}
  optionDifference =
  \begin{cases}
    x \quad \text{known} \cap y \quad \text{known} & GaussianDifference(x,y) \\
    x \quad \text{known} \oplus y \quad \text{known} & 1 \\
    x \quad \neg \text{known} \cap y \quad \neg \text{known} & 0
  \end{cases}
\end{equation}
\caption{A difference calculation used for two values ($x$ and $y$), when their values are not always known.}
\end{capeq}

Comparisons with SAR's whose chosen action was a scanning action will apply a weighted average to the health, energy, and element states of the two \texttt{AgentState}s.
Each \texttt{ElementState} within the current \texttt{AgentState} will calculate their own weighted average based on the number of the immediately adjacent cells that are known, the percent of known element values in range of the sensor, the hazard and indicator flags.
All of these \textit{elementStateDifference}s will be averaged (non-weighted) into a single difference value, \textit{averageElementStateDifference}.
This compares the usage of the element type (hazard and/or indicator detection), and knowledge of the element type (percent known within the environment).
The \textit{hazard} and \textit{indicator} differences can help the controller determine the usage of the element type's data being collected.
The \textit{percentKnown} and \textit{immediateValuesKnown} differences help the controller decide whether use of an element type's sensor is efficient or necessary.
For example, if an agent does not have knowledge of the elevation in adjacent cells, it couldn't confidently determine whether it is safe or possible to move into one of those cells without first scanning to find out.

% \begin{capeq}
%   \begin{align*}
%   \begin{equation} \label{eq:scanning_overall_difference}
%     &V = \{health_{diff},\quad energy_{diff},\quad elementStates_{diff}\} \\
%     &W = \{health_{wight},\quad energy_{wight},\quad elementStates_{wight}\} \\
%     &\\
%     &OverallDifference_{s} = WeightedAverage(V,W)
%   \end{equation}
% \end{align*}
% \caption{Calculation for the overall state difference when the compared state-action reward had chosen a scanning action, where $V$ is a list of attributes values and $W$ is the list of weights for the attributes.}
% \end{capeq}

% \begin{capeq}
% \begin{equation} \label{eq:element_state_difference}
% % \begin{align*}
%   &V = \{indicator_{diff},\quad hazard_{diff},\quad percentKnownInRange_{diff},\quad immediateKnown_{diff}\} \\
%   &W = \{indicator_{wight},\quad hazard_{wight},\quad percentKnownInRange_{wight},\quad immediateKnown_{weight}\} \\
%   &\\
%   &elementStateDifference = WeightedAverage(V,W)
% % \end{align*}
% \end{equation}
% \caption{Calculation for the overall difference of two \texttt{ElementState}s, where $V$ is a list of attributes values within the \texttt{ElementState}s and $W$ is the list of corresponding weights.}
% \end{capeq}

If the SAR's action type is movement, overall state difference is calculated using differences in each \texttt{AgentState}s' health, energy, element states, and quadrant states.
In addition to calculating \textit{elementStateDifference}s, \textit{quadrantToQuadrantDifferences} are calculated between every quadrant in the current state and every quadrant in the SAR state.
Only one ``orientation'' of quadrant-to-quadrant comparisons will be used in the overall difference calculation.
Four orientations are considered by rotating the SAR's quadrants in 90 degree intervals (see figure ~\ref{fig:quadrant_orientations}).
The resulting orientation comparisons are denoted as North-to-North, North-to-West, North-to-South and North-to-East (based on the SAR's quadrant that is matched to the current state's North quadrant).
The orientation that yields the lowest \textit{quadrantToQuadrantDifferences} (\textit{lowestQuadrantOrientationDifference}) is used in calculating $OverallDifference_{m}$.

% \begin{capeq}
% \begin{equation} \label{eq:movement_overall_difference}
% \begin{align*}
%   &V = \{health_{diff},\quad energy_{diff},\quad elementStates_{diff},\quad quadrantStates_{diff}\} \\
%   &W = \{health_{wight},\quad energy_{wight},\quad elementStates_{wight},\quad quadrantStates_{weight}\} \\
%   &\\
%   &OverallDifference_{m} = WeightedAverage(V,W)
% \end{align*}
% \end{equation}
% \caption{Calculation for the overall state difference when the compared state-action reward had chosen a movement action, where $V$ is a list of attributes values and $W$ is the list of weights for the attributes.}
% \end{capeq}

\begin{figure}[H]
  \includegraphics[width=1.0\columnwidth]{Figures/quadrant_orientations.png}
  \caption{Orientation considerations between two compared states. This displays how rotating the compared environment's quadrant orientation can reveal states of higher similarity. We see that the elevation heatmaps of the two are highly similar when compared at North-to-South orientation, where as an un-altered quadrant comparison (North-to-North) would yield a very negative comparison.}
  \label{fig:quadrant_orientations}
\end{figure}

Each orientation is important to consider because the controller is only concerned with moving towards interesting features in an environment, regardless of the direction.
Considering the orientation with the lowest difference makes the comparison relative to the two environments instead of the cardinal direction.
Consider if a highly similar SAR held information that its agent received good rewards for a particular movement action.
The current agent should be encouraged to move towards the quadrant in its own environment that holds similar features (not necessarily in the same direction).
Now, if the SAR's \textit{lowestQuadrantOrientationDifference} is found when rotating its quadrants 180 degrees (North-to-South orientation) and it had chosen to move East, the current agent should choose to move West since the two states are oriented at a 180 degree difference between each other (see figure ~\ref{fig:oriented_movement_example}).

\begin{figure}[H]
  \includegraphics[width=1.0\columnwidth]{Figures/oriented_movement_example.png}
  \caption{Example of two states that have a \textit{lowestQuadrantOrientationDifference} at North-to-South orientation. The display exemplifies how after finding the most similar orientation, the action taken by the compared state must also be re-oriented to match the \textit{lowestQuadrantOrientationDifference}. In doing so the agent will be considering movement relative to the matching features in the environment.}
  \label{fig:oriented_movement_example}
\end{figure}

Quadrant-to-quadrant comparisons produce a non-weighted average of \textit{quadrantElementStateDifference}s.
A \textit{quadrantElementStateDifference} is calculated between every \texttt{ElementState} in the current state's considered quadrant and the matching \texttt{ElementState} in the SAR state's considered quadrant (if it exists).
The \textit{quadrantElementStateDifference} uses a weighted average to compare attributes within the two \texttt{ElementState}s' considered quadrants.
For example, making a North-to-South quadrant comparison would consider element types in the current state's North quadrant against element types in the SAR's South quadrant.
When making these comparisons, it is not guaranteed that the current state and SAR state will share all the same element types.
For example, if the current state contains decibel data and the SAR state does not, no comparison can be made, and it will receive a \textit{quadrantElementStateDifference} of 1.
Because we are only comparing against element types in the current \texttt{AgentState}, if the SAR contains any element types not present in the current state, they are simply ignored.
These comparisons examine how much information about the element type is known, and the actual values in the quadrants (if they are known).
Because \texttt{averageValueDifferential} and \texttt{immediateValueDifferential} are not guaranteed to be known values for every quadrant, they use the unique option difference equation (equation ~\ref{eq:option_difference}).

% \begin{capeq}
% \begin{equation} \label{eq:quadrant_element_state_difference}
% \begin{align*}
%   &V = \{percentKnown_{diff},\quad averageValueDifferential_{diff},\quad immediateValueDifferential_{diff}\} \\
%   &W = \{percentKnown_{wight},\quad averageValueDifferential_{wight},\quad immediateValueDifferential_{wight}\} \\
%   &\\
%   &quadrantElementStateDifference = WeightedAverage(V,W)
% \end{align*}
% \end{equation}
% \caption{Calculation for comparing the difference between two \texttt{ElementState}s in given quadrants, where $V$ is a list of attributes values and $W$ is the list of weights for the attributes.}
% \end{capeq}

Once all sub-differences have been calculated and either $OverallDifference_{s}$ or $OverallDifference_{m}$ is known, the controller can decide whether the SAR qualifies to be used in future reward prediction for the current agent.
If the calculated overall difference is below the \textit{similarityThreshold}, the SAR will qualify and an instance of the \texttt{StateActionDifference} class (Appendix~\ref{appendix:state_action_difference}) is created.
Each instance stores the overall difference value, the SAR's action taken, and the short-term and long-term rewards.
State comparison will be repeated for every SAR in the memory pool, and the resulting collection of \texttt{StateActionDifference} instances is passed to the action reward prediction algorithm.



\subsection{Action Reward Prediction}
Once the controller has generated a set of \texttt{StateActionDifference}s (SAD), it will predict a short-term and long-term reward value that each possible action might receive, along with a confidence score for the predictions.
For each valid action considered, the algorithm will select a sub-set of SAD where the \texttt{StateActionDifference}'s \texttt{action} is the same as the one being considered.
Predicted short-term and long-term rewards are calculated as an average of all the \texttt{shortTermScore}s and \texttt{longTermScore}s in the sub-set.
Confidence is evaluated using the average of the \texttt{overallStateDifference}s in the sub-set, weighted by the number of \texttt{StateActionDifference}s in the sub-set (equation~\ref{eq:confidence}).
The equation will invert \textit{overallStateDifference}s when averaging them by subtracting their value from 1 (equation~\ref{eq:diff_to_similarity}).
This allows the prediction algorithm to look at them as ``similarity'' scores instead of ``difference'' scores.
If the overall difference had been 0 (the states compared were identical), their similarity score will be 1.
Because \textit{similarityThreshold} was used to filter out SARs with high overall difference values, it can be asserted that the average of all \textit{overallStateDifference}s will not fall below: $1 - similarityThreshold$.
The prediction algorithm then computes an overall \textit{actionScore} for each action using a weighted average of the predicted short-term reward, predicted long-term reward, and the confidence score.

\begin{capeq}
\begin{equation} \label{eq:diff_to_similarity}
  similarity = 1 - difference
\end{equation}
\caption{Equation for inverting an \textit{overallStateDifference} value to create a similarity value. The minimum \textit{overallStateDifference} that can exist is 0. By this logic, the highest attainable similarity between two states is 1.}
\end{capeq}

\begin{capeq}
\begin{equation} \label{eq:confidence}
  confidence = \begin{cases}
    n = 0 & 0 \\
    n < minimumSimilarStates & \frac{\sum_{n}^{i=0} 1 - SAD_{i}.overallStateDifference}{minimumSimilarStates} \\
    n >= minimumSimilarStates & \frac{\sum_{n}^{i=0} 1 - SAD_{i}.overallStateDifference}{n} \\
\end{cases}
\end{equation}
\caption{Confidence value assigned to reward prediction values based on a set of $n$ \texttt{StateActionDifference}s ($SAD$), and the $minimumSimilarStates$ value from the evolved weight set (table ~\ref{table:evolved_weight_set}).}
\end{capeq}

% \begin{capeq}
% \begin{equation} \label{eq:action_score}
% \begin{align*}
%   &V = \{predictedShortTermReward,\quad predictedLongTermReward,\quad confidence\} \\
%   &W = \{predictedShortTermReward_{weight},\quad predictedLongTermReward_{weight},\quad confidence_{weight}\} \\
%   &\\
%   &actionScore = WeightedAverage(V,W)
% \end{align*}
% \end{equation}
% \caption{Action scoring function using the action's $predictedShortTermReward$, $predictedLongTermReward$, and $confidence$, in pairing with their corresponding weights found in table ~\ref{table:evolved_weight_set}}
% \end{capeq}


\subsection{Action Selection}
Once every valid action has received an \textit{actionScore}, there are two methods the controller may use for choosing which one the agent should perform.
If the controller is being trained, roulette selection is used.
Roulette selection is an integral part of training as it will give every action a chance to be selected.
This will fill the controller memory with a variety of events both good and bad, giving the reward prediction algorithm more concise data to work with.
When the controller is being used outside of training, the action with the highest score is always selected.
Once selected, the agent will then attempt to perform the action, and its interaction with the environment will be reflected in a new \texttt{AgentState}.
If the agent is still operational after the resulting event and the goal has not yet been completed, the action decision process (figure ~\ref{fig:scout_decision_model}) will begin again using the new \texttt{AgentState}.
Once the agent is no longer operational, or the goal has been completed, the operation process ends, and new SARs are added to the controller's memory file.







% \todx{style algo}
% \begin{lstlisting}
% 1. Normalize the current state (how many SDs it falls outside of the average)
% 2. Calculate the Gaussian difference for:
%   a. health
%   b. energyLevel
%   c. elementStateDifferences = for each element state:
%     i. hazardDifference = if (current == SAR) 1 else 0
%     ii. indicatorDifference = if (current == SAR) 1 else 0
%     iii. percentKnownInSensorRangeDifference = abs(SAR - current)
%     iv. immediateValuesKnownDifference = abs(SAR - current) / 4
%   d. quadrantToQuadrantStateDifferences = for each current quadrant:
%     i. quadrantStateDifferences = for each SAR quadrant:
%       a. quadrantElementStateDifferences = for each element type:
%         i. hazardDifference = if (current == SAR) 1 else 0
%         ii. indicatorDifference = if (current == SAR) 1 else 0
%         iii. percentKnownDifference = abs(SAR - current)
%         iv. averageValueDifferentialDifference = if (current known && SAR known) abs(SAR - current) else (if current known == if SAR knonw) 1 else 0)
%         v. immediateValueDifferentialDifference = if (current known && SAR known) abs(SAR - current) else (if current known == if SAR knonw) 1 else 0)
% 3. Sum all differences together using weighted values for each Gaussian difference.
%   a. Movement state difference
%   b. Scan state difference
% \end{lstlisting}



% \subsection{Action Reward Prediction}
% Action reward prediction first calculates three factors for each valid action being considered: predicted short term reward, predicted long term reward and confidence.
% These factors are determined by StateActionDifference that hold the same action as the current action whose reward is being predicted.
% Only a list of similar StateActionDifferences are considered in these calculations.
% This list is made up of StateActionDifferences with an overall difference below a the set minDifferenceThreshold.
% Predicted short and long term rewards are calculated by the average of short and long term scores within the list of similar StateActionDifferences.
%
% If there are no StateActionDifferences in the list, the action being considered will receive a predicted short term reward and long term reward of 0.5 and a confidence of 0.


% An action's short and long term rewards are predicted from the averages of SARs where: A) the considered action was selected, and B) the state difference from the current state is below a certain threshold.
% In addition to these predicted rewards, we also calculate a confidence value for the predictions.
% \tods{confidence EQ}
% The lower the difference is between the current and SAR states, the higher the confidence will be.
% Additionally, the more SARs that are considered in the prediction, the more confident the controller can be in the predicted reward.



\chapter{Experiments and Results}
To analyze the adaptability of the SCOUt control schema, three instances of a SCOUtController are trained and then tested and tested in two experiments.
The first experiment compares the performance of each SCOUtController against a random and heuristic controller to observe if they can in fact exhibit intelligent behavior when attempting to complete a goal.
The second experiment tests the adaptability of the controllers' performances when sensors it was trained with are removed, or the goal of the agent is changed.
Two different goals are used in these experiments: Find Human and Map Water.
Find Human requires the agent to search an environment to find a Human anomaly.
Goal completion is either 100 percent for successfully locating the Human, or 0 percent for failing to find the human before depleting health or energy.
For the goal to be successfully completed, the controller must be in one of the eight adjacent cells to the Human anomaly's location in the environment. \todo{photo of human discovery zone}
Map Water tests the controller's ability to navigate an environment and collect as much data for water depth as possible.
Map Water operations will run until the entire area has been mapped, or the agent has depleted its health or energy.
Goal completion is then reward based on the percentage of the environment that was mapped for Water presence.
Training and testing are both conducted using three different EnvironmentTemplates.
Each template differs in difficulty to navigate due to the modifications present within them.

During both training and experimentation, tests are conducted to observe the performance of the controllers.
Each test will compare the intelligent SCOUtController(s) against both random and heuristic controllers.
Two heuristic controllers are used for testing with each goal.
$Heuristic_{FH}$ and $Heuristic_{MW}$ are used for Find Human and Map Water respectively.
The random and heuristic controllers will provide a base line for performance metrics.
This base line is important as the procedural generation of environments could yield conditions that vary the difficulty of achieving the goal, or even prevent goal completion entirely.
When tested against each other, all controllers are given the exact same starting conditions.
Their performances will be compared for the same goal, Environment instance and starting position.
Performance is measured in four categories: goal completion, the number of actions taken, remaining health and remaining energy.
Additionally, the starting location of the controllers for each test will be chosen in a location that does not result in damage (ex: starting in a cell with water present) or immediate goal completion (ex: starting right next to the Human).
The following sections cover the agent setup, environments used and the setup of each test and their results.


\section{Agent}
Similar Agent setup is used for every operation run in training and testing.
The only variation is in the controller being used and the sensor types present.
All health, energy, mobility and durability variables will be set to the same value throughout.
This will assure that no advantage or disadvantage is given to any controller when navigating the environment.
The performance of each controller will then be based solely on the control schema's usage of available sensors and collected environment data.
When sensors are available for an agent's use, the same instance of the sensor type is used.
The exception to this being that the indicator flag for each sensor will differ for the two goals.
For Find Human Temperature and Decibel sensors will be flagged true, and for Map Water the Water sensor will be flagged true.
Each Agent will always begin an Operation with an empty internalMap, 100.0 health and 100.0 energyLevel.
The Agent Mobility and Durabilities and each sensor instance used in experimentation are defined below.

\begin{lstlisting}[language=Scala]
 val mobility = new Mobility(
  movementSlopeUpperThreshHold: 1.0
  movementSlopeLowerThreshHold: -1.0
  movementDamageResistance: 0.0
  movementCost: 0.5
  slopeCost: 0.2
 )
\end{lstlisting}

\begin{lstlisting}[language=Scala]
 val durabilities = List(
  new Duribility (
    elementType: "Water Depth"
    damageUpperThreshold: 0.25
    damageLowerThreshold: infinity
    damageResistance: 0.0
  ),
  new Duribility (
    elementType: "Temperature"
    damageUpperThreshold: 150.0
    damageLowerThreshold: -50.0
    damageResistance: 0.0
  )
 )
\end{lstlisting}

\begin{lstlisting}[language=Scala]
 val elevationSensor = new Sensor (
   elementType: "Elevation"
   range: 30.0
   energyExpense: 0.5
   hazard: true
   indicator: Boolean
 )
\end{lstlisting}

\begin{lstlisting}[language=Scala]
 val waterSensor = new Sensor (
   elementType: "Water Depth"
   range: 1.0
   energyExpense: 1.0
   hazard: true
   indicator: Boolean
 )
\end{lstlisting}

\begin{lstlisting}[language=Scala]
 val temperatureSensor = new Sensor (
   elementType: "Temperature"
   range: 60.0
   energyExpense: 1.0
   hazard: true
   indicator: Boolean
 )
\end{lstlisting}

\begin{lstlisting}[language=Scala]
 val decibelSensor = new Sensor (
   elementType: "Decibel"
   range: 15.0
   energyExpense: 0.1
   hazard: false
   indicator: Boolean
 )
\end{lstlisting}


\section{Environment Templates}
Three EnvironmentTemplates with increasing difficulty are created for use in training and experimentation (EASY, MEDIUM, and HARD).
Increase in difficulty is achieved by increasing the environment size, adding more TerrainModifications, adjusting the average and deviation values for each ElementSeed and reducing the Effects of any Anomalies.
Increased environment size creates a wider area that an Agent will have to search or map to achieve its goal.
More TerrainModifications makes each environment potentially more hazardous.
Changing the average and deviation values of ElementSeeds used to generate the environment has a couple of effects.
First, as an example, by increasing the average decibel values in the environment, the distinction of the Human's decibel effect is dampened and the agent will need to be closer to detect the noise produced.
Also, if the variance in decibel values increases, it becomes more difficult for an agent to distinguish what consuetudes as a significant increase that may indicate the presence of the Human.
Last, by reducing the decibel effect of a Human anomaly itself, the radiation of the effect will cover less area in the environment, meaning the agent will have to search longer before it may pick up on the effect.

Each environment template has one Human that will be placed into it at random in a non-hazardous zone (so not in water).
The same templates will then be used for both the Find Human and Map Water goals.
In the case that the goal is Map Water, a Human anomaly will still be present within the environment, but it will be ignored by the agents.
Each of these EnvironmentTemplates are listed here (\todo{environment templates}) in their Json formats.
This Json format is what appears in each template's respective file.
When a template is used in training and experimentation, it will be loaded in from its file, converted to a Scala object and passed to the EnvironmentBuilder.


\section{Training}
Three separate SCOUtControllers are trained to accumulated memory pools of SAPs.
Once the training phase has completed, these controllers will be tested using their respective memory pools.
During testing, no further SAPs will be added to the controllers' memory pools.
In doing so, a consistent measure of performance can be calculated based on the memory collected during the set training phase defined here.
Each of the three controllers are trained with different goal configurations.
One is trained using the Find Human goal, the second using the Map Water goal and the third is a hybrid, trained using both goals.
They are named $SCOUt_{FH}$, $SCOUt_{MW}$ and $SCOUt_{H}$ respectively.

Each is trained for 30 iterations where an iteration runs one operation per environment template.
Once training has completed, each controller will have collected SAPs from a total of 90 operations (30 on EASY, 30 on MEDIUM and 30 on HARD).
Every operation that $SCOUt_{FH}$ is run in will be with the Find Human goal and every operation for $SCOUt_{MW}$ will be with the Map Water goal.
$SCOUt_{H}$ will alternate goals for each iterations.
In total it will have run 45 operations with the Find Human goal and 45 operations with the Map Water goal (15 on EASY, 15 on MEDIUM and 15 on HARD for both goals).

After each training iteration, the controller is tested with its current memory to track performance improvements.
Testing at each iteration runs a series of simulated operations to collect performance data.
Each series uses the controller's respective goal(s) and is run on each testing environment template 20 times (20 on EASY, 20 on MEDIUM, and 20 on HARD).
The controller tested will have access to its current memory pool that has been gathered in its training so far.
No SAPs will be gathered during these iteration tests.
For a base line, the $Random$ controller is run through the same series of tests.
Results from the 60 total iteration tests will be averaged in each of the four performance categories.
The averaged results of the learning, SCOUt controller will then be differenced against the averaged results of the $Random$ controller.
By differencing the averages, we are observing how much better or worse the SCOUt controller was able to perform than the $Random$ controller in the same testing situation.
This also removes the discrepancy between iteration tests that may have generated a series of exceptionally difficult or easy environments to work within.
It is expected that over training iterations, the goal completion, remaining health and remaining energy will increase, and the number of actions performed will decrease  in comparison to the $Random$ controller.

% The results of each SCOUt controller's training performances over time are found in \todo{training performance}.

Results for $SCOUt_{FH}$ training (figure ~\ref{fig:findhuman_training_results}) show the desired trends of increased performance over training iterations.
Average goal completion and average remaining energy begin at the same performance levels as $Random$ and increase to be consistently better than random over training.
The average number of actions performed begins slightly below $Random$ and continue to decrease before leveling out roughly two-thirds of the way through training.
This demonstrates that the controller is learning to perform more efficiently over time as both average goal completion and average remaining health show major performance boosts while fewer actions are being used.
The average remaining health of $SCOUt_{FH}$ shows slight increase over training, but for the most part is equivalent to that of the $Random$ controller.

\begin{figure}[h]
  \includegraphics[width=1.0\columnwidth]{Figures/Results/Training/SCOUt-FindHuman.JPG}
  \caption{Training performance results for $SCOUt_{FH}$}
  \label{fig:findhuman_training_results}
\end{figure}

Results for $SCOUt_{MW}$ (figure ~\ref{fig:mapwater_training_results} are less impressive.
Average goal completion and average remaining health both decrease over training, but show upward trends toward the end.
Average remaining health actually performs worse $Random$ throughout iteration testing.
$SCOUt_{MS}$ does perform well in the average number of actions taken per operation, however this is likely due to the fact that health is depleted (agent navigating into water) and the operation is ended early.
The same can be said for the remaining energy, as there will be a large amount of energy remaining after an operation is ended due to depletion of health.
The reason for these poor performance results seem to be tied with the agents inability to avoid hazardous water areas.
It is believed that the long term reward equation \todo{ref eq} did not provide adequate values for the controller to learn from because of the goal completion reward for Map Water in combination with the available energy for the agent to use.
Map Water goal completion is measured by the percent of the environment that has successfully been scanned for water.
In ~\ref{sec:experiment1}, we will see that all controllers have an average goal completion rate below 40 percent on EASY environments, below 30 percent on MEDIUM, and below 15 percent on HARD.
Poor goal completion scores will result in poor long term rewards.
The reinforcement learning schema will have difficulty making clear distinctions between "good" and "bad" actions when all SAPs in the memory pool contain low long term rewards.

\begin{figure}[h]
  \includegraphics[width=1.0\columnwidth]{Figures/Results/Training/SCOUt-MapWater.JPG}
  \caption{Training performance results for $SCOUt_{MW}$}
  \label{fig:mapwater_training_results}
\end{figure}

$SCOUt_{H}$ does not show any change in performance throughout training.
This can likely be attributed to the cancelation of good performance on Find Human mixed with poor performance on Map Water.
Iteration testing is done with both goals and results from operations with both goal types are averaged together.
Outside of average remaining health, the hybrid controller does consistently perform better than $Random$.
Performance levels are still not as high as seen in $SCOUt_{FH}$, likely due to the same issue discussed in $SCOUt_{MW}$'s results, where poor long term rewards are diluting the controller's memory pool.

\begin{figure}[h]
  \includegraphics[width=1.0\columnwidth]{Figures/Results/Training/SCOUt-Hybrid.JPG}
  \caption{Training performance results for $SCOUt_{H}$}
  \label{fig:hybrid_training_results}
\end{figure}



\section{Experiment 1} \label{sec:experiment1}
Once training has completed, the resulting SCOUt controllers are individually tested against $Random$ and respective heuristic controllers.
Test in Experiment 1 run a total of 1000 operations per environment template.
The environment template is used to generate 200 unique environments, each of which is used in 5 operations.
Performance results are averaged for each controller and compared under the three test environment templates.

\todo{Find Human}

\begin{figure}[h]
  \includegraphics[width=1.0\columnwidth]{Figures/Results/Experiment1/FindHuman.JPG}
  \caption{Performance results for $Random$, $Heuristic_{FH}$ and $SCOUt_{FH}$ on Find Human goal in different environment difficulties.}
  \label{fig:findhuman_test_results}
\end{figure}


\todo{Map Water}

\begin{figure}[h]
  \includegraphics[width=1.0\columnwidth]{Figures/Results/Experiment1/MapWater.JPG}
  \caption{Performance results for $Random$, $Heuristic_{MW}$ and $SCOUt_{MW}$ on Map Water goal in different environment difficulties.}
  \label{fig:mapwater_test_results}
\end{figure}


\todo{Hybrid}

\begin{figure}[h]
  \includegraphics[width=1.0\columnwidth]{Figures/Results/Experiment1/HybridFindHuman.JPG}
  \caption{Performance results for $Random$, $Heuristic_{FH}$ and $SCOUt_{H}$ on Find Human goal in different environment difficulties.}
  \label{fig:hybrid_findhuman_test_results}
\end{figure}

\begin{figure}[h]
  \includegraphics[width=1.0\columnwidth]{Figures/Results/Experiment1/HybridMapWater.JPG}
  \caption{Performance results for $Random$, $Heuristic_{MW}$ and $SCOUt_{H}$ on Map Water goal in different environment difficulties.}
  \label{fig:hybrid_mapwater_test_results}
\end{figure}





\section{Experiment 2} \label{sec:experiment2}
The second experiment is broken into three tests: goal changing, sensor set changing and additional training.

Goal changing tests the performance of $SCOUt_{FH}$ on the Map Water goal and the performance of $SCOUt_{MW}$ on the Find Human goal.
The SCOUt controllers will have no training on the new goal they are tested within.
This will exhibit their ability to complete a goal using only memory from the original goal they were trained on.



\chapter{Conclusion} \label{ch:conclusion}
The wide variety of use cases for autonomous robotics in the field of exploration suggests the need of a unified solution for the setup and execution of related operations.
Unique environments and tasks are found throughout the problem spaces of exploration that often require a robotic agent to be constructed from the ground up.
SCOUt provides a platform for both modeling and simulating wide varieties of goal driven tasks within an environment, as well as an adaptive control solution for robotic agents.
 % that can adjust its behaviors based on the available set of surveillance tools it has to work with.
The abstracted data structures used by SCOUt offer a framework that can easily be utilized and expanded upon by the growing communities of robotics and exploration.

Simulation testing is a valuable tool for planning out exploration-based operations.
It offers both cost and risk avoidance solutions for building, training, and testing new ideas.
SCOUt's simulation platform touches all of these features in the form of a user friendly setup tool.
New testing scenarios can be created with minimal input allowing a user to direct their focus towards the primary tasks at hand.
In this project, thousands of operations were simulated with a variety agents, environments, goals, and the interactions between them.
Each operation holds the opportunity to present a unique scenario for each controller to be tested within.
Features of the controlled agent can be adjusted to reflect its available sensors, maneuverability, and durability to environmental factors.
Environments are procedurally generated to produce unique features within similar settings, and agent starting positions are chosen randomly within.
Data collected from all of these operations could then be averaged, charted, and analyzed to track the performance of different controllers across the vast problem space.
The simulation platform also allowed multiple controllers to be tested and compared in identical conditions, removing discrepancies between each unique operations faced.
% This allows performances to be compared independently from the operation setups that are generated.

The SCOUt project aimed to uncover the concept of a generalized work flow that lies within exploration in unknown environments.
Through research in a wide variety of studies, I found the core components of such operations to be a continuous cycle of gathering and analyzing data to draw new beliefs and conclusions that help to progress a goal.
My memory-based learning control schema presented in this paper demonstrates a unified solution built on this premise.
Data from both the environment and the agent is condensed into a single ``state'' that is passed as input to the decision model.
The model will then decide if actions should be performed to gather more data to analyze, or if the controller should navigate the agent towards new, potentially interesting areas within the environment.
Both types of actions will work together in the cycle of state analysis and decision making to progress goal completion.
Two types of exploration based goals were examined in varying environmental conditions: anomaly searching and element type mapping.
SCOUt's unique control schema was tested for its performance and adaptability in these two types of scenarios.
Heuristic control schemas were also built for each of the specific goals, and were tested in tandem to the SCOUt's control model.
The heuristic controllers follow the same process of state analysis and action decisions, but apply logical analyses rather than a prediction model based on memory of past events.
Heuristic controllers created are focused on completing one specific goal and reflect a specialized, non-adaptive control solution that might be applied in these scenarios.

It is seen in my results that SCOUt's model was in fact able to perform adaptively across a variety of situations.
SCOUt demonstrated the ability to learn task related behaviors that could be applied to multiple goals.
When changing the goal, available sensors, and environmental settings, SCOUt was able to maintain an efficient level of performance.
In scenarios such as the removal of sensors, some controllers showed little to no drop in performance at all.
In the majority of performance categories, the memory-based learning controllers showed superior results compared to that of the heuristic controllers.
With better average rates of goal completion, number of actions taken, and remaining energy, we see how SCOUt's model is both adaptive \textit{and} efficient.
Even in some cases where SCOUt is placed in scenarios in which it was not trained, we still see results that are more efficient than the heuristic approaches.
On these grounds, it is strongly believed that autonomous control can be abstracted into a unified solution for exploration related operations.

One area of the memory-based control schema that could use improvement is hazard avoidance.
In the \textit{Map Water} operations especially, we see SCOUt controllers suffer in their ability to maintain the agent's health.
SCOUt's analytical ability relied heavily on a dense network of weights tied to examining past rewards it had received for performing actions.
Both the weight and reward systems create a grey area that likely caused poor performance results in hazard avoidance.
A proper level of importance in the agent's remaining health was not being reflected in the decision model.
This is a common caveat seen when artificial intelligence (AI) is left to independently control every aspect of a task.
When AI is introduced to real-world environments, there are certain aspects of problems that typically have desired behaviors which seem trivial from a human's instinctual base of knowledge.
However, it cannot be guaranteed that an AI will pick up on these since they can only ``think'' analytically and not instinctually.
When facing undesirable behavior, AI solutions are often enhanced using sets of ``rules'' that they must follow.
For example, when building autonomous self-driving vehicles, rules are often embedded into the vehicle's control schema to assist with safety (stay within a designated lane, always drive at a safe distance behind the vehicle in front of you, etc.).
Applying similar sets of rules to SCOUt's control schema could have greatly improved its performance in hazard avoidance and subsequently in all other areas of performance that were measured.
This project elected to forgo any hard coded rules into SCOUt's control schema for two reasons.
First, all testing and training was done in simulation so there were no real-world risks involved in a ``rogue'' AI approach to control.
Second, I wanted to test the memory-based learning model to the fullest of its capabilities without the assistance of any external knowledge.

This project opens research potentials into similar abstracted approaches within the fields of autonomous robotics and exploration.
Results suggest the potential for other categories of autonomics that could be abstracted into their own unified control models.
Top-down approaches could be applied by finding the underlying work flow of each sub-field to generalize the process and reduce the amount of repetitive work required in finding solutions on a case-by-case basis.




\section{Future Work} \label{sec:future_work}
This section covers a few ideas for future work that could improve upon the current state of the SCOUt project.

\subsection{Behavior Rules}
Adding a set of hard-coded rules to SCOUt's decision model would likely lead to even better results.
Ideally, I would want to prevent the controller from selecting actions that would damage the agent or have no benefit to the operation (e.g., using sensors in areas that have already been mapped or moving into quadrants that are already mapped).
Each time the controller is given a set of valid actions to choose from, they could be passed through the set of rules and any undesirable actions could be removed from consideration.
Using a rule set would additionally provide a tool for investigating suspected causes of poor performance.
In my results, we saw a poor performance in remaining health and I suspected that the controller was not properly learning to avoid hazards.
Rules for avoiding movement into cells with water or large drops in elevation could each be implemented and tested independently.
Testing each rule in isolation would help to determine if poor health performances were being caused by one or the other, by neither, or by both.
Rules which show obvious performance improvements could then be implemented by the controller to prevent undesired behaviors.

\subsection{Artificial Neural Network Integration}
Original ideas for the adaptive control schema included the use of an artificial neural network (ANN).
The ANN would take an agent state and a list of valid actions as input and output the selected action.
Due to the dynamic nature of agent states, use of an ANN was ruled out.
If an agent changes the sensors it is equipped with, the ANN would need to account for a new set of input values due to the different set of \texttt{ElementState}s that would be found in the \texttt{AgentState}.
However, the memory-based approach required the use of several weights to guide each state comparison and action scoring equation.
My solution was to optimize the weight set using a genetic algorithm (GA).
Use of an ANN in place of the state comparison equations would eliminate the need for external weights to be provided.
This could further enhance the adaptability of SCOUt's decision model.
State-action rewards (SARs) could be compared against the current state by passing attributes through the ANN to output difference scores.
The ANN could be trained with backpropagation using the existing simulation platform and the short-term and long term-reward systems.
This model would likely require the ANN to remain ``open'' in the sense that its weights are constantly being trained during every operation.

\subsection{Improved Memory Management}
Improvements to both the process of saving and loading the SCOUt controller's memory could be made.
For saving memory, the application of cluster would cut down on memory storage requirements.
Currently, SCOUt saves the last 20 SARs from each operation, and uniformly samples 5 percent of the remaining SARs.
Instead of only saving a sub-set, all SARs could be saved in memory and a data clustering algorithm could later be applied to ``clean-up'' the memory.
The clusters could then be averaged and given a weight based on the number of SARs that fell within the cluster.
This would eliminate redundancies within the memory set, as well as reduce the amount of computational time required for the controller to conduct state comparisons.
As for loading memory, an adaptive approach could be taken to select only a sub-set of the SARs to use in each operation.
The controller could begin by analyzing the given goal and agent setup to choose SARs from memory that correlate to the operation at hand.
I expect that this would have two positive effects: less memory means less computational time for the action decision model, and a more concise memory set would yield higher performance results.

\subsection{Integration of Goals into Agent States}
Currently, the only place that we see the reflection of the current goal in SCOUt's decision model is with the \texttt{indicator} flag found in each \texttt{ElementState}.
A higher-level approach could be taken so that the decision model analyzes an \textit{OperationState} rather than just an \texttt{AgentState}.
This could be achieved by classifying the goal type and goal instance.
For example, a \textit{Find Human} \textit{OperationState} would also include a current goal type of ``anomaly searching'' and a goal instance of ``human'' since this is the specific anomaly that the agent is searching for.
These attributes could then be weighted into the state comparison system to produce a more concise \textit{OverallStateDifference} score.

% 

\chapter{Future Work}
How to improve this approach.

Dynamic environments


Sensors can scan in an arc rather than 360 degrees.


Better Memory selection/ trimming.


\bibliographystyle{plain}
\bibliography{References}



% Redefine Headings
\renewcommand{\thechapter}{Appendix}
\titleformat{\chapter}{\normalfont\bfseries\centering}{\thechapter}{.25cm}{\uppercasetitle}[]

\renewcommand{\thesection}{\thechapter\enspace\Alph{section}}
\titleformat{\section}{\normalfont\bfseries\justifyheading}{\thesection.}{6pt}{}[]

% Prevent auto-add to TOC. Have to manually add to look nice.
\newcommand{\nocontentsline}[3]{}
\newcommand{\tocless}[2]{\bgroup\let\addcontentsline=\nocontentsline#1{#2}\egroup}
\addtocontents{toc}{\protect\setcounter{tocdepth}{1}}


\chapter*{APPENDICES} \label{appendix}
\addcontentsline{toc}{chapter}{Appendices}

\setcounter{section}{0}

\stepcounter{appxcounter}
\pagebreak
\tocless\section{Environment Data Structures} \label{sec:environment_data_structures}
\addcontentsline{toc}{section}{\thesection. Environment Data Structures}
This appendix contains all data structures related to an environment.
Together these traits, classes, and instances can be combined in different ways to create unique models of real-world environments.

\begin{appxlst}
  \caption{A class for representing a real-world environment. It is laid out in a grid of cells that contain information related to each of their respective areas within the grid.}
  \lstinputlisting{Codes/Environment.scala} \label{appendix:environment_class}
\end{appxlst}


\begin{appxlst}
  \caption{A class that represents a sub-section of an environment. The data stored in each instance represents the features that are found within the \texttt{Cell}'s area within the environment.}
  \lstinputlisting{Codes/Cell.scala} \label{appendix:cell_class}
\end{appxlst}


\begin{appxlst}
  \caption{An \texttt{Element} trait is a generalized representation of a measurable feature type within an environment. Specific element types can be created by extending this trait. Each instance defines specific information about what values can be represented for the specific element type.}
  \lstinputlisting{Codes/Element.scala} \label{appendix:elelment_trait}
\end{appxlst}


\begin{appxlst}
  \caption{\texttt{Elevation} is a class which extends the \texttt{Element} trait. This class models elevation levels within an environment. Different instances can be created to specify a level of elevation for each cell in an environment.}
  \lstinputlisting{Codes/Elevation.scala} \label{appendix:elevation_class}
\end{appxlst}


\begin{appxlst}
  \caption{An \texttt{Anomaly} is any object within an environment that may be of interest. Anomalies often have a set of effects that will alter the environment around them. This trait can be extended to define specific types of anomalies that can be represented in an environment.}
  \lstinputlisting{Codes/Anomaly.scala} \label{appendix:anomaly_trait}
\end{appxlst}

\begin{appxlst}
  \caption{The \texttt{Effect} trait is a generalized description of an alteration that an \texttt{Anomaly} has on the environment. Effects will alter a single element type in an area of the environment that the anomaly is located within.}
  \lstinputlisting{Codes/Effect.scala} \label{appendix:effect_trait}
\end{appxlst}

\begin{appxlst}
  \caption{\texttt{Human} is a specific \texttt{Anomaly} class. This class represents a peron that could be found in an environment. Human's have two defined \texttt{Effects}: sound and heat. These effects will alter the decibel and temperature element values in the human's general area within the environment.}
  \lstinputlisting{Codes/Human.scala} \label{appendix:human_class}
\end{appxlst}

\begin{appxlst}
  \caption{A \texttt{Layer} holds a collection of instances of a specific \texttt{Element} class. The collection is represented as a 2-dimentional grid that is relative to an \texttt{Environment} grid. They can be thought of as the same structure as an environment, but only containing information about a single element type.}
  \lstinputlisting{Codes/Layer.scala} \label{appendix:layer_class}
\end{appxlst}





\stepcounter{appxcounter}
\pagebreak
\tocless\section{Agent Data Structures} \label{sec:agent_data_structres}
\addcontentsline{toc}{section}{\thesection. Agent Data Structures}
This appendix contains data structures that are representative of an agent.
These structures model the different capabilities an agent has to interact with an environment.


\begin{appxlst}
  \caption{An \texttt{Agent} represents a physical member capable of acting within an environment. The class defines a controller for selecting actions, sensors that the agent is equipped with, mobility and durability features of the agent for modeling interactions with an environment, and several internal status variables.}
  \lstinputlisting{Codes/Agent.scala} \label{appendix:agent_class}
\end{appxlst}


\begin{appxlst}
  \caption{A \texttt{Sensor} is a tool that an agent can utilize to collect data about a specific element type within an environment. Sensors have a set range and energy cost related to using them.}
  \lstinputlisting{Codes/Sensor.scala} \label{appendix:sensor_class}
\end{appxlst}


\begin{appxlst}
  \caption{\texttt{Mobility} contains a set of variables related to how an agent will be able to safely move within an environment.}
  \lstinputlisting{Codes/Mobility.scala} \label{appendix:mobility_class}
\end{appxlst}


\begin{appxlst}
  \caption{\texttt{Durability} defines how an agent will interact with an element type in an environment. Different agents will each have strengths and weaknesses defined by how they will react when in contact with certain elements in an environment.}
  \lstinputlisting{Codes/Durability.scala} \label{appendix:durability_class}
\end{appxlst}


\begin{appxlst}
  \caption{An \texttt{AgentState} represents an instance of the internal status of an agent and the information that the agent knows about its environment. An \texttt{Agent}'s internal map is condensed into a set of sub-states within the entire agent state for each element type that the agent has knowledge of.}
  \lstinputlisting{Codes/AgentState.scala} \label{appendix:agentstate_class}
\end{appxlst}


\begin{appxlst}
  \caption{\texttt{ElementState}s are representative of an agent's knowledge of a specific element type in an environment. The information contained is directly related to what the agent has gathered into its internal map through the use a sensor. Specific known values of the element type are divided into four quadrants relative to the agent's current position.}
  \lstinputlisting{Codes/ElementState.scala} \label{appendix:elementstate_class}
\end{appxlst}


\begin{appxlst}
  \caption{A \texttt{QuadrantState} represents a collection of known information about a specific element type in a sub-set of cells within an environment. The data is condensed to an average known value differential and immediate known value differential relative to the value in the cell that the agent currently occupies.}
  \lstinputlisting{Codes/QuadrantState.scala} \label{appendix:quadrantstate_class}
\end{appxlst}


\begin{appxlst}
  \caption{\texttt{Controller}s are the decision-making models that are used to select actions for an agent to take. The process that each controller uses to select an action vary, but must choose from a set of valid actions and can use information that the agent has gathered while exploring the environment.}
  \lstinputlisting{Codes/Controller.scala} \label{appendix:controller_trait}
\end{appxlst}





\stepcounter{appxcounter}
\pagebreak
\tocless\section{Environment Generation Data Structures} \label{sec:environment_generation_data_structures}
\addcontentsline{toc}{section}{\thesection. Environment Generation Data Structures}
This appendix includes the data structures used to guide the procedural generation of unique environments.

\begin{appxlst}
  \caption{\texttt{EnvironmentTemplate}s hold the entire collection of attributes that are used to guide the process of generating an environment.}
\lstinputlisting{Codes/EnvironmentTemplate.scala} \label{appendix:environmenttemplate_class}
\end{appxlst}


\begin{appxlst}
  \caption{\texttt{ElementSeed} is a trait used to define helper classes for specific \texttt{Element} classes. They have a set of attributes that can be set to guide how the element type will be initialized in an environment and functions related to the actual process in which the element type will be procedurally generated within the environment.}
\lstinputlisting{Codes/ElementSeed.scala} \label{appendix:elementseed_trait}
\end{appxlst}


\begin{appxlst}
  \caption{A \texttt{TerrainModification} is a trait used for defining processes to alter element types within the environment. These help to add unique features within element types found in the environment.}
\lstinputlisting{Codes/TerrainModification.scala} \label{appendix:terrainmodification_trait}
\end{appxlst}





\stepcounter{appxcounter}
\pagebreak
\tocless\section{State Comparison Data Structures} \label{sec:state_comparison_data_structures}
\addcontentsline{toc}{section}{\thesection. State Comparison Data Structures}
This appendix includes data structures related to comparing multiple \texttt{AgentState}s to each other.

\begin{appxlst}
  \caption{\texttt{GaussianData} holds the mean and standard deviation of a collection of values.}
\lstinputlisting{Codes/GaussianData.scala} \label{appendix:gaussian_data}
\end{appxlst}


\begin{appxlst}
  \caption{\texttt{StateActionDifference} holds values related to a comparison that was made between a current \texttt{AgentState} and an \texttt{AgentState} within SCOUt's memory of state-action rewards. Each instance defines the differences that were calculated between two states, an overall state difference, the action that was taken by the agent in the state-action reward, and then the short-term and long-term rewards that the were received for the action.}
\lstinputlisting{Codes/StateActionDifference.scala} \label{appendix:state_action_difference}
\end{appxlst}








\stepcounter{appxcounter}
\pagebreak
\tocless\section{Experimentation Setup} \label{sec:experiment_setup}
\addcontentsline{toc}{section}{\thesection. Experimentation Setup}
This appendix includes the code listings for the \texttt{Agent} configuration and \texttt{EnvironmentTemplate}s used in experimentation.

\begin{appxlst}
  \caption{Instances of the \texttt{Agent} class and \texttt{Sensor} classes that are used in experimentation. Some attributes are set per operation during experiments. These are marked with a comment ``Defined Per Operation.''}
  \lstinputlisting{Codes/TestAgentSetup.scala} \label{appendix:agent_setup}
\end{appxlst}

\begin{appxlst}
  \caption{JSON data storing the an environment template used in experimentation. This template is title \textit{EASY} as it represents a relatively easy environment for an operation to take place in. The environment is $8\times8$ cells and contains only one modification for a pool of water to be generated within the environment.}
  \lstinputlisting[language=Java]{Codes/EASY.json} \label{appendix:easy_environmenttemplate}
\end{appxlst}

\begin{appxlst}
  \caption{JSON data storing the an environment template used in experimentation. This template is title \textit{MEDIUM} as it presents a few challenging features that an agent may face. The environment is $10\times10$ cells and contains both a ``hill'' elevation modification and a water pool modification. Compared to the \textit{EASY} environment template, \textit{MEDIUM} has higher ambient levels of decibel and temperature values, making it slightly more difficult to identify the effects of a human anomaly within the environment.}
  \lstinputlisting[language=Java]{Codes/MEDIUM.json} \label{appendix:medium_environmenttemplate}
\end{appxlst}

\begin{appxlst}
  \caption{JSON data storing the an environment template used in experimentation. This template is title \textit{HARD} as it presents many challenging features that an agent may face. The environment is $12\times12$ cells and contains a ``hill'' elevation modification, a ``valley'' elevation modification, a water pool modification, and a water stream modification. Additionally, the ambient levels of decibel values and temperature values are raised and the sound and heat effects of the human anomaly are suppressed even more than seen in the \textit{MEDIUM} environment template. The combination of all of these factors create environments that are both highly difficult to safely navigate and difficult to identify anomaly effects within.}
  \lstinputlisting[language=Java]{Codes/HARD.json} \label{appendix:hard_environmenttemplate}
\end{appxlst}






\stepcounter{appxcounter}
\pagebreak
\tocless\section{Training Variation 1} \label{sec:training_variation1}
\addcontentsline{toc}{section}{\thesection. Training Variation 1}
This appendix contains results for our first variation of training the three SCOUt controller memories ($SCOUt_{FH}$, $SCOUt_{MW}$, and $SCOUt_{H}$).
During training and iteration testing, the long-term reward (algorithm~\ref{algorithmic:long_term_reward}) was adjusted to only factor in a \texttt{goalReward} if the agent had remaining health at the end of an operation.
This was done to observe how SCOUt controllers would change the agent's behavior when their memory of state-action rewards reflected smaller long-term rewards from operations where the agent's health was depleted.
It was hoped that we would see better performance in average remaining energy and subsequently all other areas.
However, we found no significant improvements in performance.
Appendix~\ref{appendix:findhuman_training_variation1} shows the results for $SCOUt_{FH}$, Appendix~\ref{appendix:mapwater_training_variation1} shows the results for $SCOUt_{MW}$, and Appendix~\ref{appendix:hybrid_training_fh_variation1} and~\ref{appendix:hybrid_training_mw_variation1} show results for $SCOUt_{H}$ in \textit{Find Human} and \textit{Map Water} operations, respectively.

\begin{appxfig}[H]
\begin{figure}[H]
  \includegraphics[width=0.9\columnwidth]{Figures/Results/TrainingVariation1/FindHuman.JPG}
\end{figure}
\caption{Iteration testing performance results for $SCOUt_{FH}$ attempting \textit{Find Human} using setup variation 1 (see subsection~\ref{subsec:training_variations}). All graphs show the controller's average difference in performance compared to $Random$ ($SCOUt_{FH}$ average - $Random$ average) VS the number of training iterations completed.}
\label{appendix:findhuman_training_variation1}
\end{appxfig}


\begin{appxfig}[H]
\begin{figure}[H]
  \includegraphics[width=0.9\columnwidth]{Figures/Results/TrainingVariation1/MapWater.JPG}
\end{figure}
\caption{Iteration testing performance results for $SCOUt_{MW}$ attempting \textit{Map Water} using setup variation 1 (see subsection~\ref{subsec:training_variations}). All graphs show the controller's average difference in performance compared to $Random$ ($SCOUt_{MW}$ average - $Random$ average) VS the number of training iterations completed.}
\label{appendix:mapwater_training_variation1}
\end{appxfig}


\begin{appxfig}[H]
\begin{figure}[H]
  \includegraphics[width=0.9\columnwidth]{Figures/Results/TrainingVariation1/Hybrid-FindHuman.JPG}
\end{figure}
\caption{Iteration testing performance results for $SCOUt_{H}$ attempting \textit{Find Human} using setup variation 1 (see subsection~\ref{subsec:training_variations}). All graphs show the controller's average difference in performance compared to $Random$ ($SCOUt_{H}$ average - $Random$ average) VS the number of training iterations completed.}
\label{appendix:hybrid_training_fh_variation1}
\end{appxfig}


\begin{appxfig}[H]
\begin{figure}[H]
  \includegraphics[width=0.9\columnwidth]{Figures/Results/TrainingVariation1/Hybrid-MapWater.JPG}
\end{figure}
\caption{Iteration testing performance results for $SCOUt_{H}$ attempting \textit{Map Water} using setup variation 1 (see subsection~\ref{subsec:training_variations}). All graphs show the controller's average difference in performance compared to $Random$ ($SCOUt_{H}$ average - $Random$ average) VS the number of training iterations completed.}
\label{appendix:hybrid_training_mw_variation1}
\end{appxfig}







\stepcounter{appxcounter}
\pagebreak
\tocless\section{Training Variation 2} \label{sec:training_variation2}
\addcontentsline{toc}{section}{\thesection. Training Variation 2}
This appendix contains results for our second variation of training the three SCOUt controller memories ($SCOUt_{FH}$, $SCOUt_{MW}$, and $SCOUt_{H}$).
During training and iteration testing, the \textit{goalRewardWeight} used for calculating long-term reward (algorithm~\ref{algorithmic:long_term_reward}) was set to 1.5.
This was done to observe how SCOUt controllers would change the agent's behavior when their memory of state-action rewards reflected a stronger emphasis on the level of goal completion attained.
It was hoped that we would see better performance in all categories.
While only slight improvements were observed, this method was chosen for all testing conducted in our experimentation.
Appendix~\ref{appendix:findhuman_training_variation2} shows the results for $SCOUt_{FH}$, Appendix~\ref{appendix:mapwater_training_variation2} shows the results for $SCOUt_{MW}$, and Appendix~\ref{appendix:hybrid_training_fh_variation2} and~\ref{appendix:hybrid_training_mw_variation2} show results for $SCOUt_{H}$ in \textit{Find Human} and \textit{Map Water} operations, respectively.

\begin{appxfig}[H]
\begin{figure}[H]
  \includegraphics[width=0.9\columnwidth]{Figures/Results/TrainingVariation2/FindHuman.JPG}
\end{figure}
\caption{Iteration testing performance results for $SCOUt_{FH}$ attempting \textit{Find Human} using setup variation 2 (see subsection~\ref{subsec:training_variations}). All graphs show the controller's average difference in performance compared to $Random$ ($SCOUt_{FH}$ average - $Random$ average) VS the number of training iterations completed.}
\label{appendix:findhuman_training_variation2}
\end{appxfig}


\begin{appxfig}[H]
\begin{figure}[H]
  \includegraphics[width=0.9\columnwidth]{Figures/Results/TrainingVariation2/MapWater.JPG}
\end{figure}
\caption{Iteration testing performance results for $SCOUt_{MW}$ attempting \textit{Map Water} using setup variation 2 (see subsection~\ref{subsec:training_variations}). All graphs show the controller's average difference in performance compared to $Random$ ($SCOUt_{MW}$ average - $Random$ average) VS the number of training iterations completed.}
\label{appendix:mapwater_training_variation2}
\end{appxfig}


\begin{appxfig}[H]
\begin{figure}[H]
  \includegraphics[width=0.9\columnwidth]{Figures/Results/TrainingVariation2/Hybrid-FindHuman.JPG}
\end{figure}
\caption{Iteration testing performance results for $SCOUt_{H}$ attempting \textit{Find Human} using setup variation 2 (see subsection~\ref{subsec:training_variations}). All graphs show the controller's average difference in performance compared to $Random$ ($SCOUt_{H}$ average - $Random$ average) VS the number of training iterations completed.}
\label{appendix:hybrid_training_fh_variation2}
\end{appxfig}


\begin{appxfig}[H]
\begin{figure}[H]
  \includegraphics[width=0.9\columnwidth]{Figures/Results/TrainingVariation2/Hybrid-MapWater.JPG}
\end{figure}
\caption{Iteration testing performance results for $SCOUt_{H}$ attempting \textit{Map Water} using setup variation 2 (see subsection~\ref{subsec:training_variations}). All graphs show the controller's average difference in performance compared to $Random$ ($SCOUt_{H}$ average - $Random$ average) VS the number of training iterations completed.}
\label{appendix:hybrid_training_mw_variation2}
\end{appxfig}





%
% \tocless\section{Appendix C: Mathematical Proofs} \label{appendix:proofs}
% This appendix contains mathematical proofs related to the use of Gaussian normalization on value sets.
%
%
%
% \begin{algorithm}[H]
%   \setstretch{1.35}
%   \caption{Proof that...}
%   \begin{algorithmic} \label{appendix:gaussian_difference_identical}
%     \REQUIRE $mean \leftarrow 10$
%     \REQUIRE $sd \leftarrow 1$
%     \REQUIRE $x = y$
%     \ENSURE $x_{normal} = y_{normal}$
%     \STATE $x_{normal} \leftarrow (x - mean) / sd$
%     \STATE $x = y$
%     \STATE $x = y$
%     \STATE $x = y$
%     \STATE $x = y$
%     \STATE $x = y$
%
%   \end{algorithmic}
% \end{algorithm}
%
%
% \begin{lstlisting}[label=appendix:gaussian_difference_identical]
% Example:
% Gaussian mean = 10
% Gaussian standard deviation = 1
%
% x = 12
% y = 12
% x = y
%
% x(normalized) = (x - Gaussain mean) / Gaussian standard deviation
% x(normalized) = (12 - 10) / 1
% x(normalized) = 2 / 1
% x(normalized) = 2
%
% y(normalized) = (y - Gaussain mean) / Gaussian standard deviation
% y(normalized) = (12 - 10) / 1
% y(normalized) = 2 / 1
% y(normalized) = 2
%
% x(normalized) = y(normalized)
%
% Gaussian difference (x,y) = |x - y|
% Gaussian difference (x,y) = |2 - 2|
% Gaussian difference (x,y) = 0
% \end{lstlisting}
%
%
% \begin{lstlisting}[label=appendix:gaussian_difference_different]
% Example:
% Gaussian mean = 10
% Gaussian standard deviation = 1
%
% x = 12
% y = 7
%
% x(normalized) = (x - Gaussain mean) / Gaussian standard deviation
% x(normalized) = (12 - 10) / 1
% x(normalized) = 2 / 1
% x(normalized) = 2
%
% y(normalized) = (y - Gaussain mean) / Gaussian standard deviation
% y(normalized) = (7 - 10) / 1
% y(normalized) = -3 / 1
% y(normalized) = -3
%
% Gaussian difference (x,y) = |x - y|
% Gaussian difference (x,y) = |2 - -3|
% Gaussian difference (x,y) = 5
% \end{lstlisting}


\end{document}
