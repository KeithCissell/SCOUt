\doublespacing

\chapter*{INTRODUCTION}
\addcontentsline{toc}{chapter}{Introduction}
As research in the fields of autonomous systems and robotics have become more extensive, it is evident that there are a wide range of application for robots with integrated autonomy.
There are rovers, drones and even aquatic robots capable of intelligently selecting actions to perform in their environments.
The tasks that these robots carry out greatly vary based on the robot's abilities and the environmental limitations.
This variance causes a demand for distinct software and hardware that is required to achieve each robot's given task.
Many similarities can be drawn from the wide rang of robots and their usage.
Almost all autonomous robots operate through their use of observational sensors to gather data and a control schema to analyze the data collected and plan actions.

A great deal of research has been done in hybrid robots which use adaptive hardware that is multifunctional to various tasks.
However, there is not an extensive amount of research on software with the capability to integrate with multiple robot compositions and tasks.
Most of this is due to the fact that each robot has unique capabilities that do not overlap with many other robots.
Autonomous robots tend to focus in on a certain niche that require their systems to be built from the ground up for each task presented.
This leaves the question of what pieces of autonomous control can be abstracted.

There are many intelligent approaches that can be applied to decision making processes.
The field of autonomous robotics has benefited tremendously through the growing field of artificial intelligence (AI).
AI methods are commonly used in situations when there is a known number of controllable variables and a wide solution space to be explored.
This makes them great candidates for creating a system which drives the decision-making process of an autonomous robots.
In particular, neural networks and reinforcement learning architectures trained in simulations have yielded promising results for finding optimal control patterns in the diverse applications of autonomous robots.

The Surveillance Coordination and Operations Utility (SCOUt) system takes a top down approach to create an adaptive control schema for diverse robotic agents and their uses by abstracting the very basics of autonomous robotics.
This control schema repeatedly follows the process of collecting of data from sensors, analyzing the agent's state, and the outputting response controls to complete surveillance based operations.
SCOUt uses memory based reinforcement learning to make state based action decisions.
The abstraction of this process allows the collected memory pool to be used adaptively across varying environments and robot setups.

% This is where you begin your thesis. Be sure to indent all paragraphs 1/2 inch. Do not put extra spacing between paragraphs. In addition, paragraphs should be formatted so that there is zero added space above/below them. Justify on the left side only. Consult the style manual approved by the faculty in your program to determine appropriate actions on widow and orphan issues \cite{article-minimal}.
%
% \section{Spacing}
% Double space the body of the text (except the abstract). Single spacing may be required for footnotes or quotations of five lines or more, and may be used for table headings and figure captions. In addition, single spacing is acceptable for subheadings in the Table of Contents and in the Acknowledgement Page if this enables these sections to be one page. Finally, references may be single spaced within the reference and double spaced between references. Consult the style manual approved by the faculty in your program for appropriate reference format. Also, note that after the primary heading on this page, and before each secondary heading, there are blank lines. This is appropriate spacing format for these types of headings.
%
% \section{Headings}
% Headings are essential for dividing the body of the thesis, and a standard format is required by the Graduate College. This format may be an exception to the style manual approved by the faculty in your program, but you are to follow the Thesis Guide.
%
% Headings should be descriptive, focus attention on distinctive sections, and thus enable a quick targeting of salient information addressed in the thesis. Depending on the nature of the subject, more than one level of heading may be appropriate. It is vital that there is a consistency in placement and other aspects of formatting headings that divide the text.
%
% Start main (primary) headings on a new page. These primary headings should be centered, bold, upper case, and separated from the text that follows by extra space (blank line). Secondary headings should follow a blank line (double-spacing), be placed at the left margin, bold, and capitalize only the first letter of words. Tertiary headings will be placed as the first word(s) of the paragraph of that section, indented, bolded, first letter of words capitalized, and followed by a period. Fourth level headings should be indented, underlined, with the first letter of words capitalized, followed by a period; fifth level heading should be indented, italicized with the first letter of words capitalized, followed by a period. The first sentence of the paragraph will then follow on the same line for 3 rd - 5 th level headings. The format of primary, secondary, and tertiary headings is modeled in this Thesis Guide. Make sure your Table of Contents matches these heading types.
%
% It is not acceptable to have just one subheading under a larger heading. For example, if you are to use secondary headings under a primary heading, there must be two or more headings. This would be analogous to an outline that has an ``A'' but no ``B.''
